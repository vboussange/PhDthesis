% !TEX root = ../../my-thesis.tex
% \newcommand{\M}{\mathcal{M}}
\newcommand{\modnull}{\M_{null}}
\newcommand{\modalphap}{\M_{\alpha^+}}
\newcommand{\modalphan}{\M_{\alpha^-}}
\newcommand{\modmu}{\M_\mu}
\newcommand{\moddelta}{\M_\delta}
\newcommand{\hc}{^{(c)}}
\newcommand{\LL}{\mathcal{L}}
\newcommand{\BIC}{\text{BIC}}
% \newcommand{\by}{\textbf{y}}

\graphicspath{{./content/chap4_econobiology/figures/}}

\chapter{Analogous processes to ecological interactions and dispersal shape the dynamics of economic activities}
\label{chap:econobiology}

by Victor Boussange$^{1,2}$,
Didier Sornette,
Heike Lischke$^{3}$,
and 
Loïc Pellissier$^{1,2}$\bigskip
\\
$^1$ \small{Swiss Federal Research Institute WSL, CH-8903 Birmensdorf, Switzerland}\smallskip\\
$^2$ \small{Landscape Ecology, Institute of Terrestrial Ecosystems, Department of Environmental System Science, ETH Zürich, CH-8092 Zürich, Switzerland}\smallskip\\
$^3$ \small{Dynamic Macroecology, Land Change Science, Swiss Federal Research Institute WSL, Birmensdorf, Switzerland}\smallskip
% $^1${Email: \href{mailto:bvictor@ethz.ch}{\texttt{bvictor@ethz.ch}}}\smallskip\\
% $^2${Email: \href{mailto:loic.pellissier@usys.ethz.ch}{\texttt{loic.pellissier@usys.ethz.ch}}}\smallskip\\
% \smallskip

\fancyabstractQuote{
Thus, dare! your heritage, your acquisition, \\
Stories, lessons from the mouths of your fathers, \\
Laws and customs, names of the ancient Gods, \\
Forget them with hardiness to raise your eyes \\
like newborns, on divine nature.}{Friedrich Hölderlin (1770-1843)}{The Death of Empedocles (1st version)}{
  The fundamental processes of ecological interactions, dispersal and mutations shape the dynamics of biological systems, and analogous eco-evolutionary processes acting upon firms have been proposed to explain economic change. This hypothesis is compelling because it explains economic change through endogenous mechanisms, but whether it can characterize economic change at the country level is unclear.
  % 
  Here, we use an inverse modelling technique, together with 59 years of economic data covering 77 countries, to test whether the collective dynamics of economic activities at the country level can be characterized by eco-evolutionary processes.
  % 
  We estimate the statistical support of population dynamic models coupling the dynamics of economic activities by accounting, respectively, for positive and negative interactions between themselves, their spatial dispersal, and their transformations into other economic activities. 
  % 
  We compare the models' statistical support against a logistic growth model taken as a null model, and find that the null model is rejected in 45 countries against the alternative models. In particular, we find a strong support for the models capturing positive interactions between economic activities, and spatial dispersal across countries. These results suggest that processes akin to those displayed in ecosystems play a significant role on the dynamics of economic systems at the national level.
  % 
  The strength-of-evidence obtained for each model varies across countries, and might underlie differences in proximities between countries, or specific institutional contexts and historical contingencies.
  % 
  Overall, our study provides a new quantitative, biologically inspired framework to study the forces shaping economic change.
}

  \section{Introduction}

  %%%%%%%%%%%%%%%%%%%%%%%%%%%%%%%%%%%%%%%%%%%%%%%%%%%%%%%%%%
  %%%%% First paragraph economics / physics / biology %%%%%%
  %%%%%%%%%%%%%%%%%%%%%%%%%%%%%%%%%%%%%%%%%%%%%%%%%%%%%%%%%%
  The fields of evolutionary biology and economics have mutually exchanged ideas for the past two centuries \citep{Dopfer2007}, and analogies between processes in biology and economics have been increasingly used during the 20th century to explain economic change \citep{Ruth1996}.
  % 
  A pioneer was Schumpeter, who notoriously proposed that economic dynamics is driven by innovations that transform the economy -- so called periods of "creative destruction" \citep{schumpeter2017theory}-- similar to the punctuated equilibrium changes observed throughout the development of life on Earth \citep{gould1972}.
  %
  The analogy with biology was further developed within the field of evolutionary economics \citep{Hodgson2019}, largely promoted by the seminal work of \cite{nelson1985evolutionary}.
  % 
  The premise of evolutionary economics considers habits, customs, and organizational routines as "replicators", i.e. atomic units playing the role of genes and defining as a whole the cohesive identity of an economic entity \citep{Hodgson2019}.
  % 
  Under this framework, firm business strategies determine how they transform commodities and knowledge into new knowledge, new technologies or other industrial products with more added value.
  % 
  The replicators define the fitness of an economic entity within a given economic context, which grows in terms of assets and human capital and survive within an ecosystem of entities \citep{Hodgson2002}, continuously adapting and experiencing evolutionary processes \citep{Veblen1898}.
  % 
  While biological analogies have resulted in useful insights on the plausible drivers of economic growth \citep{Dopfer2007}, the proposed pathways have mostly been qualitatively investigated.
  % 
  Computational tools have recently allowed to test biological hypotheses against data to gain a quantitative understanding of the ecological and evolutionary processes shaping the dynamics of ecological systems \citep{Pontarp2019,Boussange2022a,Skeels2022}. These tools could leverage the qualitative insights gained from biological analogies and provide a quantitative framework to investigate the forces shaping economic change.
  % \todo[inline]{econobiology discussed in this article \citep{Rosser2010}}
  
  %%%%%%%%%%%%%%%%%%%%%%%%%%%%%%%%%%%%%%%%%%%%%%%%%%%%%%%%%%
  %%%%%                 processes biology / economics %%%%%%
  %%%%%%%%%%%%%%%%%%%%%%%%%%%%%%%%%%%%%%%%%%%%%%%%%%%%%%%%%%
  Interactions between biological organisms, dispersal (movement of individuals across space), and mutations, represent fundamental processes that drive the dynamics of ecosystems \citep{Vellend2010}, and similar eco-evolutionary processes may shape the dynamic of economic systems.
  %
  Analogously to biological organisms, economic entities interact in a mutualistic or competitive fashion \citep{Pistorius1997}. While biological organisms engage in negative interactions when e.g.~competing for similar resources \citep{GRIME1973}, economic entities have negative effects on each other when relying on workforce with similar knowledge or attracting similar potential investors \citep{Wernerfelt1989}.
  % 
  Positive interactions are observed between biological organisms with e.g the accumulation of nutrients, the provision of shade, or the protection from herbivores \citep{Callaway2002}. In parallel, positive effects between economic entities appear when they are connected through supply chains \citep{Ozman2009,Saavedra2009a}, or when they benefit from positive agglomeration externalities \citep{VanDerPanne2004}, such as knowledge spillovers \citep{Caragliu2016} or the attraction of elite workers \citep{Cohendet2018}.
  %%
  Dispersal processes play a major role in the development of ecosystems, where e.g. colonization initiate ecological succession with the settlement and growth of new species \citep{Leibold2004}. Similarly, socio-economic processes contribute to diffuse knowledge and organizational routines across space. Those processes involve, e.g., international business expansions \citep{Andersen1993,Zahra2000,Luo2007}, labor mobility \citep{Boschma2008}, or the diffusion of innovations through social networks \citep{RogersEverettM2003DoI,Keller2004,Bahar2014a}.
  %   
  %   international business expansions are orchestrated by firms to increase profitability \citep{Zahra2000} and reach new markets \citep{Luo2007} and workforce. Other self-organized .
  %%
  Finally, evolutionary processes allow the adaptation of organisms to changing environmental conditions \citep{Bell2017}, while in economic systems, variations in organizational routines allow economic entities to adapt to economic contexts \citep{Cordes2006} and transform into new economic entities \citep{Freeman2002,Hodgson2004,Aldrich2008}. %An example of evolutionary processes is the innovations in the oil, gas and synthetic material industry that lead to the motorization of transport, civil economy and war, and to the mechanisation of agriculture \citep{Freeman2002}. 
  %  
  While processes acting upon economic activities, including their interactions, their spatial dispersal, and their transformations into other economic activities, are regularly documented in evolutionary economics, the significance of these processes on long-run economic change has seldom been quantified.%\todo{Talk about diffusion in the product space: \citep{Hidalgo2021}}

  %%%%%%%%%%%%%%%%%%%%%%%%%%%%%%%%%%%%%%%%%%%%%%%%%%%%%%%%%%
  %%%%% Literature review on quantitative models      %%%%%%
  %%%%%%%%%%%%%%%%%%%%%%%%%%%%%%%%%%%%%%%%%%%%%%%%%%%%%%%%%%
  Focusing on endogenous forces and borrowing concepts and methods from biology, a number of modelling approaches have broken with the traditions of standard economic modelling \citep{10.1093/cje/bet027} to study the processes driving economic patterns \citep{Tacchella2018}.
  % 
  For instance, \cite{Saavedra2009a} investigated the effect of cooperation within firms with a model of consumer-resource interactions. \cite{Scholl2020} developed a theory of market ecology to interpret market phenomena and predict market behavior, where financial trading strategies are the analogous to biological species.
  % 
  Lotka-Volterra models were used in the prediction of technology evolution \citep{Zhang2018}, to evaluate product and firm competition \citep{Modis1997,Saavedra2014}, to understand the drivers of market share dynamics \citep{Farmer1999,Michalakelis2011,Marasco2016,Gatabazi2019} and to estimate the value of firms \citep{Cauwels56}.
  % 
  \cite{Applegate2021} investigated an ecological model of competition-colonization dynamics to understand the distribution of firm sizes, and \cite{Suweis2015} used a population dynamic model to evaluate the effect of international trade on global food security.
  % 
%   Nonetheless, the confrontation of the model predictions has remained mostly qualitative, in that patterns rather than full model outputs have been compared with observations.
  % 
  The availability of fine-grained datasets of economic activities, together with dimensionality reduction techniques, have provided insights into the endogenous processes shaping economic development \citep{Mealy2019,Hidalgo2021}.
  % 
  In particular, the detailed global trade data compiled by the United Nations statistical Division and cleaned in \cite{Hidalgo2021} consists in a time series of 59 years of economic activity that can be combined with inverse modelling techniques to learn about the processes influencing the long-run dynamics of economic systems.
  
  %%%%%%%%%%%%%%%%%%%%%%%%%%%%%%%%%%%%%%%%%%%%%%%%%%%%%%%%%%
  %%%%% 4th paragraph : results                       %%%%%%
  %%%%%%%%%%%%%%%%%%%%%%%%%%%%%%%%%%%%%%%%%%%%%%%%%%%%%%%%%%
  Here, we investigate whether eco-evolutionary processes can quantitatively explain economic growth using an inverse modelling technique together with 59 years of data on economic activity. Relating the temporal evolution of the capital of economic activities to the temporal evolution of the biomass of functional groups within an ecosystem, we quantitatively estimate the effect of eco-evolutionary processes on the long-run development of national economies.
  % 
  Functional groups, as the aggregation of species sharing similar characteristics and having similar functions within the ecosystem, are the required elemental units to model ecosystem dynamics. Analogously, as the aggregation of firms with similar output nature \citep{Applegate2021}, we consider 9 economic activities to form the elemental units of economic systems, and conduct our investigation by modelling their temporal dynamics.
  % 
  We consider a null model implementing the fundamental processes of self-replication and self-limitation ($\modnull$), where no couplings between economic activities are captured, which we contrast to alternative dynamic models coupling their dynamics, including negative and positive ecological interactions between themselves ($\modalphan,\modalphap$), their spatial dispersal ($\moddelta$), and their transformations into other economic activities ($\modmu$).
  % 
  We use the mini-batching machine learning technique of \cite{Boussange2022a} to estimate the maximum likelihood of each model for 100 countries, and use a model selection technique to evaluate the statistical support of each alternative model, based on a data set of national exports from 1962 to 2020, taken as a proxy for the temporal development of the capital of economic activities. 
  %
  We first show that the machine learning method, together with the model selection procedure, can provide support for the generating processes in a controlled experiment.
  %   
  Further, applying the model selection procedure to the empirical data, we find evidences for eco-evolutionary processes, and observe differences in model support across countries, which we relate to socio-economic drivers. 
  %%
  By adopting a biological perspective on economic dynamics, our study sets the basis for an understanding of the endogenous forces determining economic growth, and proposes a complementary perspective to mainstream economic theory on the drivers of economic dynamics.

%%%%%%%%%%%%%%%%%%%%%%%%%%%%%%%%%%%%%%%%%%%%%%%%%%%%%%%%%%%%%%
%%%%%%%%%%%%%% METHODS    %%%%%%%%%%%%%%%%%%%%%%%%%%%%%%%%%%%%
%%%%%%%%%%%%%%%%%%%%%%%%%%%%%%%%%%%%%%%%%%%%%%%%%%%%%%%%%%%%%%   

\section{Methods}\label{sec:methods_econobio}
\subsection{Eco-evolutionary model to characterize the dynamics of economic activities}

We derive a general population dynamic model where the dynamics of national economic activities is driven by self-replication, self-limitation, interactions between themselves, their spatial dispersal across countries, and their transformations into other economic activities within the country. We present the general model in the following, which we further split into alternative sub-models to test the support of each process.

In the general population dynamic model that includes all the processes investigated, we consider that the rate of change of the size of an economic activity $i$ in the country $c$, denoted as $n_i\hc $, is determined as follows
%; see \citep{Nordbotten} for a variant model where populations are structured over a continuous space.

\begin{equation}\label{eq:model_general}
  \begin{split}
    \tfrac{d}{dt} n_i\hc(t)  &= r_i\hc  n_i\hc(t) \left(1 -   b_i\hc  n_i\hc(t)  + \sum_{j \neq i}^{N\hc} \alpha\hc _{i,j} n_j\hc(t) \right) + \sum_l^M \delta^{(l,c)}_{i} \left( n_i^{(l)}(t) - n_i\hc(t)  \right) \\
    &\qquad + \sum_j^{N\hc} \mu\hc _{j,i}\left( n_j\hc(t)  - n_i\hc(t)  \right),
  \end{split}
\end{equation}
where $N\hc$ is the number of economic activities considered in country $c$ and $M$ is the number of countries considered.
% 
In \cref{eq:model_general}, the first summand corresponds to a Lotka-Volterra model for ecological communities (see, e.g., \cite{Bunin2017,Scheffer2006a,Case1990}), where  $r_i\hc $ is the growth rate, involved in self-replication, and $b_i\hc $ is the self-interaction coefficient, involved in self-limitation. $b_i\hc $ can also be interpreted as the inverse of the country's carrying capacity for the activity. %\citep{Grimme}
%
The individual dynamics of economic activities are altered by interactions, where $\alpha_{i,j}\hc $ captures the interaction between activity $ i $ and activity $ j $ at location $c$. The interaction between $i$ and $j$ is mutualistic if $\alpha_{i,j}\hc  > 0$, and competitive when $\alpha_{i,j}\hc  < 0$, altering positively or negatively the rate of change of the activity $i$ by the presence of other activities.
%
The second summand accounts for spatial dispersal across locations (see, e.g., \cite{Tilman1994b}), where $\delta^{(l,c)}_{i}$ is the rate of spatial dispersal for activity $i$ between locations $l$ and $c$.
% 
The last summand corresponds to economic activity transformations within country $c$, where $\mu\hc _{j,i}$ is the rate of transfer dictating how fast activity $j$ transforms into activity $i$. In the sense of the quasi-species model \citep{eigen1988molecular}, this term can be interpreted as accounting for evolutionary processes, and can contribute to the development of an activity in the presence of activity $j$.
%
A graphical representation of the model is given in \cref{fig:model}.

%%
\Cref{eq:model_general} involves a high number of independent parameters to fit (i.e., $N\hc(1 + M + 2N\hc)$ parameters for each country), which can be reduced for the sake of parsimony under mean field assumptions. Specifically, we assume that all economic activities interact similarly, so that $\alpha_{i,j}\hc  = \alpha\hc $. We assume that activity transformations are symmetric and occur at similar rates for all activities, so that $\mu_{i,j}\hc  = \mu\hc $, and we assume that spatial dispersal occur at similar rates for all activities and all countries, so that $\delta^{(l,c)} _{i} = \delta^{(c)}$. These mean field assumptions reduce the number of parameters to $2N+3$. 
% 
In summary, the parameters $r_i\hc$ and $b_i\hc$ are assumed activity- and country-dependent, and $\alpha\hc$, $\mu\hc$, and $\delta\hc$ are assumed country-dependent. We further assume that the values of these parameters are determined by the activity's characteristics and the country's institutional system (comprising e.g.~the taxation regime, the system of innovation, the legal system, the intellectual property right and the socio-cultural background of the society) and resources (comprising e.g.~labor force, knowledge capital, agricultural resources, mineral resources and energy resources), but that they do not change through time.

%%
To investigate how the data supports each of the process embedded in \cref{eq:model_general}, we further decompose the model into five different sub-models, as follows

  \begin{align}
  \label{eq:submodels1}
    \M_{null}&:  &&\tfrac{d}{dt} n_i\hc(t)  = r_i\hc  n_i\hc(t) (1 -  b_i\hc n_i\hc (t) ) \\
  \label{eq:submodels2}
    \modalphap&: && \tfrac{d}{dt} n_i\hc(t)  = r_i\hc  n_i\hc(t) \left(1 -  b_i\hc n_i\hc(t) + \alpha\hc  \sum_{j \neq i} n_j\hc(t) \right), \quad \alpha > 0\\
      \label{eq:submodels3}
    \modalphan&: && \tfrac{d}{dt} n_i\hc(t)  = r_i\hc  n_i\hc(t) \left(1 -  b_i\hc  n_i\hc(t) + \alpha\hc  \sum_{j \neq i} n_j\hc(t) \right), \quad \alpha < 0\\
      \label{eq:submodels4}
    \moddelta&: && \tfrac{d}{dt} n_i\hc(t)  = r_i\hc  n_i\hc(t) (1 - b_i\hc  n_i\hc(t) ) + \delta\hc \left(\overline{n_i\hc}(t) - n_i\hc(t) \right) \\
      \label{eq:submodels5}
    \M_{\mu}&: && \tfrac{d}{dt} n_i\hc(t)  = r_i\hc  n_i\hc(t) (1 -  b_i\hc n_i\hc(t) ) + \mu\hc \sum_j^{N\hc} \left( n_j\hc  (t) - n_i\hc(t)  \right)
  \end{align}

% Mathematical properties of equivalent model have theoretically been investigated in 
% \begin{itemize}
%     \item 
%         \citep{Rohr2014}: structural stability = dynamically stable and feasible equilibrium. It is shown that an interaction matrix that is Lyapunov-diagonally stable grants the global stability of any potential feasible equilibrium.
%     \item 
%         ??
% \end{itemize}
where 
\begin{equation}\label{eq:n_overline}
      \overline{n_i\hc } = \frac{1}{M-1} \sum_{l \neq c}^M n_{i}^{(l)}
\end{equation}
accounts for the capital of activity $i$ at the global level, excluding country $c$. The term $\overline{n_i\hc }$ appearing in \cref{eq:submodels5,eq:n_overline} arises from the mean field assumption used for spatial dispersal, where 

\begin{equation}
  \begin{split}
    \sum_l^M \delta^{(l,c)}_{i} \left( n_i^{(l)}(t) - n_i\hc(t)  \right) &= \delta\hc  \sum_l^M  \left(n_i^{(l)}(t) - n_i\hc(t) \right) \\
    &= \delta\hc \sum_{l \neq c}^M \left(n_i^{(l)}(t) - n_i\hc(t) \right) \\
    &= \delta\hc \left( \sum_{l \neq c}^M n_i^{(l)}(t) - (M - 1) n_i\hc \right) \\
    &\propto \delta\hc \left( \overline{n_i\hc } - n_i\hc \right).
  \end{split}
\end{equation}

%%
The simplest model $\modnull$ embeds self-replication and self-limitation without any further coupling force acting upon economic activities, and is considered as the null model.
% 
The alternative models $\modalphap$, $\modalphan$, $\moddelta$, and $\modmu$ capture self-replication and self-limitation together with one of the additional eco-evolutionary process investigated.
% 
Because we find inconsistent maximum likelihood estimations in the controlled experiment detailed in the \nameref{sec:results-minibatch} for models incorporating combinations of eco-evolutionary processes, we do not test more complex models.
% 
For $\moddelta$, in \cref{eq:submodels4}, we assume that spatial dispersal do not affect the dynamics of $\overline{n_i\hc}$. As such, the dynamics of $\overline{n_i\hc}$ does not depend on $n_i\hc$, and we use linear interpolations of the data to model $\overline{n_i\hc}$. Dividing by two the number of state variables to simulate in each country, this assumption considerably improves the computational complexity of the maximum likelihood estimation procedure.
% 
In the following, we designate by $\M(t,\theta\hc )$ the vector of economic activity capitals predicted by model $\M$ at time $t$ for country $c$ with the parameter vector $\theta\hc$, that includes the growth rates $r_i\hc $, the self limitation rates $b_i\hc$, the specific model parameters $\alpha\hc$, $\mu\hc$, $\delta\hc$, and the initial condition vector $n\hc (t_0)$ for country $c$.

% The model is structurally identifiable (i.e parameters can be infered from time series) \citep{Remien2021}. 


\begin{figure}[ht]
  % \includegraphics[width=0.8\textwidth]{img/conceptual.png}
  \includegraphics[width=0.8\textwidth]{conceptual/model_tikz.pdf}
  \centering
  \caption{\small
      \textbf{General population dynamic model used to characterize the dynamics of economic activities, capturing interactions between themselves, their spatial dispersal and their transformations into other economic activities.}
      % 
      For each country $c$, each one of the 9 economic activities considered is assumed to be driven by replication and self limitation processes, characterized respectively by a growth rate $r_i\hc$ and a self-limitation coefficient $b_i\hc$, which values are assumed to be country-dependent and determined by the institutional context and the resources available (in terms of human capital, knowledge capital and natural resources). 
      % 
      In addition, interactions between activities within each country are determined by $\alpha\hc$ and can be mutualistic ($\alpha\hc > 0$) or competitive ($\alpha\hc< 0$).
      % 
      Due to spatial dispersal, the capital of an economic activity can experience positive (resp. negative) fluxes of capital when it is lower (resp. higher) than the average global capital for the economic activity at time $t$, arising at rate $\delta\hc$.
      % 
      Due to transformations into other economic activities, the capital of an economic activity can experience positive (resp. negative) fluxes of capital when it is lower (resp. higher) than the capital of other economic sectors within the country at time $t$, arising at rate $\mu\hc$.
      % 
      The parameters $r_i\hc$, $b_i\hc$, $\alpha\hc$, $\delta\hc$ and $\mu\hc$ that best fit the empirical data are estimated in each country for each sub-model in \cref{eq:submodels1,eq:submodels2,eq:submodels3,eq:submodels4,eq:submodels5}, to investigate whether the proposed processes have a significant effect on the temporal dynamics of economic activities. This diagram has been designed using resources from Flaticon.com.
      }\label{fig:model}
\end{figure}
\FloatBarrier
    
\subsection{Empirical data and model likelihood}

We consider time series of global trade data as a proxy for the evolution of the capital of economic activities through time. Export data is synthetic of the capital of an economic activity within a country because the more it is exporting, the more it is competing in the global market, and therefore the larger it is in terms of capital \citep{Tacchella2018}. We specifically use the database compiled by the United Nations Statistical Division COMTRADE, categorized in the Standard International Trade Classification (SITC, revision 2) at the 1 digit level, covering 11 categories of economic activities in 249 countries from 1962 to 2020. Because of their low quality \citep{Hidalgo2021}, we discard the "Unspecified" and "Services" category, resulting in 9 economic categories to model (see \cref{fig:fits} for details). As shown in the \nameref{sec:results-minibatch}, this large number of data points contains the information necessary to recover the plausible generating processes.
%%
Export values for activity $i$ in country $c$, denoted by $X_{i}\hc (t)$, are discounted by the national population at time $t$, denoted by $P\hc (t)$, in order to compare the economic variables across time and across countries. This leads to expressing the observation data for the capital of activity $i$ in country $c$ as 
% 
$ y_i\hc (t) = X_{i}\hc (t) / P\hc (t) $.
% 
%although RCA is usually further scaled by the ensemble export product within the country as the end goal of complexity economics is not to characterise growth but specialisation of an economy.
%%
In each country, we only consider economic activities which have sustained a significant size relative to the world trends for at least 4 years (revealed comparative advantage $>1$, see \cite{Hidalgo2021}).
% 
We assume that the observation data is contaminated by a white noise with log-normal distribution $\epsilon$ with zero mean and a variance-covariance matrix $\Sigma = \sigma^2 I$ where $\sigma$ is the noise level, which is a reasonable error model for population dynamics \citep{Schartau2017}, so that
% 
$y\hc _i(t)  = n\hc _i(t) \exp(\epsilon_i\hc (t))$.
% 
This leads to expressing the likelihood of model $\M$ in country $c$, denoted by $\LL(\theta\hc_\M | {\by\hc }, \M)$, as

\begin{equation}\label{eq:likelihood_lognormal}
  \LL(\theta\hc_\M | {\by\hc }, \M) = \prod_{j=1}^{T\hc} \frac{1}{\sqrt{(2\pi)^{N\hc}|\Sigma|} y\hc(t_j)} \exp \left(-\frac{1}{2} \Big[ \tilde{d}\hc(t_j)\Big]^{T\hc} \Sigma^{-1} \Big[ \tilde{d}\hc(t_j)\Big] \right)
\end{equation}
% 
where $\theta\hc_\M$ refers to the parameters and the initial conditions for model $\M$, $\by\hc = (y\hc(t_1), \allowbreak \dots, y\hc(t_T))$  designate the $T\hc$ time points of economic data available for country $c$, $ y\hc(t) = (y_1(t), \dots, y\hc_N(t_T))$ is the vector of economic activity capital at time $t$, and $\tilde{d}\hc(t_j) = \ln(y\hc(t_j)) - \ln(\M(t_j,\theta\hc_\M))$. 
% 
We detail in the next section how do we obtain the maximum likelihood estimate $\hat\theta\hc_\M$ that maximizes \cref{eq:likelihood_lognormal} given $\by\hc$ for each model $\M$ and country $c$, further used for selecting the most probable model given the data \citep{Burnham2002}. In the following, we drop the indices $\hc$ for clarification.

\subsection{Maximum likelihood estimation}\label{parameter-inference}

In order to obtain the maximum likelihood of each model given the data, we employ the machine learning framework detailed in \cite{Boussange2022a}, which is based on a variational optimization method and a learning strategy coined "mini-batching". The likelihood of model $\M_i$ is maximized by training $\M_i$ against mini-batches of data comprising only $K < T\hc$ data points of the full time series, where the parameters and the initial conditions for each segment are estimated.
% 
The mini-batching strategy ensures the convergence towards the maximum likelihood estimate, provided that the choice of $K$ is appropriate given the data and the model investigated \citep{Boussange2022a}. 
% 
A large $K$ might induce convergence towards a local minimum, while a low $K$ might flatten the likelihood landscape, where all models would be provided equal support. 
% 
In the controlled experiments in the \nameref{sec:results-minibatch}, we find that $K=20$ (models are trained against batches of 20 years) ensures a good convergence, while reasonably discriminating the models (\cref{fig:synthetic_test_all_AIC}). We therefore use $K=20$ throughout all the experiments detailed in the manuscript, and discard countries where data is available for less than $K = 20$ years.
% 
The library \textbf{PiecewiseInference.jl} \citep{Boussange2022a} is employed for the numerical implementation, and we use the gradient descent optimizer ADAM \citep{Kingma2014} during the first 800 epochs to converge in the basin of attraction of the maximum likelihood estimate, which we substitute for the Broyden-Fletcher-Goldfarb-Shanno optimizer (BFGS) \citep{fletcher2013practical} for the final 800 training epochs to ensure faster and more accurate convergence.
% 
As a cross-checking procedure, for each country and each model we perform 5 optimization runs with different initial parameter values, where $r_i$, $b_i$, $\alpha$, $\mu$, $\delta$ are drawn from the uniform random distribution $\mathcal{U}_{[0.05, 0.15]}$, $\mathcal{U}_{[0.5,1.5]}$, $\mathcal{U}_{[0.5,1.5]}$, $\mathcal{U}_{[0.0005, 0.0015]}$ and $\mathcal{U}_{[0.0005, 0.0015]}$ respectively.
% 
We then take the best run among the 5 optimization runs, making sure that the likelihood estimates of each run are similar (see \cref{tab:results} for full details of the results).
% 
% We used a gradient descent optimization method to minimise $L(\theta\hc )$. We used the library \textbf{PiecewiseInference.jl} to train the ODE parameters $\theta^{(c)_\M}$ with the gradient descent optimiser ADAM and BFGS. Both the parameters and the intitial conditions for each activities were fitted.

% \todo[inline]{how structural change that leads to increasing output variety was gradually perceived by economists and eventually incorporated into models of economic growth: \citep{Saviotti2020}}    
%%%%%%%%%%%%%%%%%%%%%%%%%%%%%%%%%%%%%%%%%%%%%%%%%%%%%%%%%%%%%%
%%%%%%%%%%%%%% RESULTS    %%%%%%%%%%%%%%%%%%%%%%%%%%%%%%%%%%%%
%%%%%%%%%%%%%%%%%%%%%%%%%%%%%%%%%%%%%%%%%%%%%%%%%%%%%%%%%%%%%%


\section{Results}
\label{sec:results-minibatch}
\subsection{Validation with synthetic data}\label{sec:synthetic}
We first investigate in a controlled experiment whether the inverse modelling technique of \cite{Boussange2022a}, together with the proposed sub-models in \cref{eq:submodels1,eq:submodels2,eq:submodels3,eq:submodels4,eq:submodels5}, can detect signatures of eco-evolutionary processes in a controlled experiment.
%
We proceed by generating multiple synthetic datasets from the models $\modalphan$, $\modalphap$, $\M_\delta$ and $\modmu$ with realistic $r_i$ and $b_i$ parameters and with different values for the parameters $\alpha$, $\mu$, and $\delta$ and with different values for the noise level $\sigma$. We consider $\modnull$, $\modalphan$, $\modalphap$, $\M_\delta$ and $\modmu$ as equally plausible candidate models for each generated datasets, and apply the maximum likelihood estimation method detailed in the \nameref{sec:methods_econobio} to obtain the maximum likelihood $\LL(\hat \theta, \M | \by)$ of each model for each generated dataset. We then use the Bayesian information criterion (BIC) to select the model with the strongest strength-of-evidence in relation to the data \citep{Mangan2017}, calculating the BIC for model $\M_i$ as $\BIC_{\M_i} = -2 \ln(\LL(\hat \theta, \M_i | \by)) + k_{\M} \ln(N T ) $
% 
where $N$ is the number of activities in country $c$, $T$ is the number of time points in the time series considered ($N T$ therefore being the number of data points for country $c$) and $k_{\M}$ is the number of parameters in the model $\M$ for country $c$.
% 
The BIC ranks the most probable models by penalizing complexity to balance information loss and parsimony, where candidate models with the lowest scores are ranked as the most likely \citep{Mangan2017}.
% 
We consider the relative BIC score $\Delta \BIC_{\M_i}$, which allows a strength-of-evidence comparison across models and is calculated as 
%
$\Delta \BIC_{\M_i} = \BIC_{\M_i} - \min_{j} \BIC_{\M_j}$. 
%
We expect the $\Delta \BIC$ scores to only provide support for the true generating model when the process considered has a significant effect on the observed dynamics.
%%
Under realistic observational noise ($\sigma = 0.2$), we find overall strong support for the true models (i.e., when $\M_i$ is the true generating model, $\Delta\BIC_{\M_{i}} = 0$ and $\Delta\BIC_{\M_{j,j\neq i}} > 10$, see \cref{fig:synthetic_test_all_AIC}\textbf{A-C}) when the values of the parameters $\alpha$, $\mu$ and $\delta$ exceed a certain threshold. 
% 
% In particular, our results indicate that when $\Delta \BIC_{\M_i} = 0 $ and when for all the other models $\M_j$, $j\neq i$, we have that $\Delta\BIC_{\M_j} > 10$, then $\M_i$ is systematically the true model.
% 
% This is in accordance with classical model selection criterion rules \citep{Burgnam}, where it is assumed that a difference in $\BIC$ greater than 10 between two models guarantees support for the model with the lower $\BIC$.
% 
Moreover, the more extreme the value of $\alpha$, $\mu$, and $\delta$, the more support is given to the true generating model (\cref{fig:synthetic_test_all_AIC}\textbf{A-C}).
% 
In contrast, when the signature of the underlying process on the dynamics is not sufficiently strong, the null model $\modnull$ is given the most strength-of-evidence ($\Delta\BIC_{\modnull} = 0)$.
% 
% On the other hand, $\Delta \BIC_{\modnull} < 10$ indicates that the signature of the underlying processes on the dynamics is not strong enough, and in this case, the parsimony principle should favor the least complex model, the null model $\modnull$.
% 
These results hold for varying noise levels (see \cref{figSI:synthetic_test_all_AIC} for $\sigma = 0.3$), indicating that the models together with the maximum likelihood estimation method are well adapted to investigate the influence of eco-evolutionary processes with the dataset considered.
%
Based upon this experiment and classical model selection criterion \citep{Burnham2002}, we decide to accept the hypothesis that the null model is the best model given the data when $\Delta \BIC_{\modnull} \leq 10$. In addition, when $\BIC_{\M_i} - \BIC_{\modnull} < -10$ we conclude that $\M_i$ is supported against $\modnull$, and we conclude that $\M_i$ is the best model given the data if $\Delta \BIC_{\M_i} = 0 $ and if for all other model $\M_{j,j \neq i}$, we have that $\Delta \BIC_{\M_{j,j \neq i}} > 10$.
%%
Overall, the proposed eco-evolutionary model together with the maximum likelihood estimation leads to a good discrimination ability when combined with the BIC-based model selection procedure, and can provide strength-of-evidence for eco-evolutionary processes that may shape the dynamics of economic systems.

\begin{figure}
  \centering
  \includegraphics{synthetic_test_all_AIC.png}
  \caption{\small
  \textbf{Validation of the model selection procedure.} In \textbf{A} and $\textbf{B}$ we generate data with models $\modalphan$ and $\modalphap$ with varying $\alpha$, in \textbf{C} we generate data with model $\moddelta$ with varying $\delta$ and in \textbf{D} we generate data with model $\modmu$ with varying $\mu$. We then contaminate the generated data with realistic noise ($\sigma = 0.2$), and seek for the maximum likelihood of all models $\modnull$, $\modalphan$, $\modalphap$, $\moddelta$ and $\modmu$. The maximum likelihood of each model is then used to compute difference in BIC for each model, $\Delta \BIC$, which comparison provides a strength-of-evidence for each model. 
  %   
  \textbf{A}-\textbf{D} show that the model selection procedure is valid, since only the true models are placed in the "Strong support" category when no other model is given support (equivalently, when all other model are placed in the "No support" category).}
  \label{fig:synthetic_test_all_AIC}
\end{figure}
\FloatBarrier

\subsection{Discrepancies in dynamical regime across countries}

We apply the maximum likelihood estimation method detailed in the \nameref{sec:methods_econobio} to obtain the maximum likelihood of each model for the world's richest 100 countries (highest GDP per capita as of 2020, see list in \cref{tab:country_codes}). 
% 
We first investigate the quality of best fit, and further detail the results for each model in the following section. %as it is assumed that the corresponding data is less fluctuating \citep{Tacchella2018}.
% 
In the following, we refer to countries by their ISO 3166-1 reference code, which associated plain names are given in \cref{tab:country_codes}.
% 
Details of the numerical simulations are provided in \cref{tab:results}.
%
Over the 100 countries investigated, only 78 countries have sufficiently long time series to be included in the analyses (countries with at least 20 years of data, see \nameref{sec:methods_econobio}), while 77 countries show consistent maximum likelihood estimates (best model coefficient of determination $R^2$ > 0).
% 
Among the 77 countries, we find that the best models have a good fit to the data, and capture well the long-run growth of economic activities (median explained variance $R^2 = 0.940$, see \cref{fig:fits} and \cref{figSI:fit_alphan,figSI:fit_alphap,figSI:fit_delta,figSI:fit_mu} for graphical illustrations of the fit).
% 
While this demonstrates the relevance of the proposed models, we observe discrepancies across countries in how well the best models perform (standard deviation of explained variance std$(R^2)= 0.074$, see \cref{fig:fits}\textbf{D} for a graphical illustration of the variance in terms of the model loglikelihoods).
% 
% the loglikelihood of a country is a very good predictor of the logarithm of its current GDP (\cref{fig:fits}\textbf{D} and \cref{table:gdp_loglikelihood}, $R^2 = 0.65$), and explain this strong association by two reasons.
% 
This discrepancy is explained by the number of data points available in each country, as it differs among countries and since more data points yield a higher model loglikelihood ($\beta = 0.928 \pm 0.046$, $p<0.001$, see \cref{table:gdp_loglikelihood}).
% 
We additionally observe a positive effect of the GDP on the residuals ($\beta = 0.693 \pm 0.187$, $p<0.001$, see \cref{table:gdp_loglikelihood} and \cref{fig:fits}\textbf{D}), indicating that rich economies have a more deterministic behavior than poorer ones. The residuals are further negatively associated with the number of economic activities ($\beta = -0.230 \pm 0.071$, $p<0.01$, see \cref{table:gdp_loglikelihood}), indicating that the models proposed might be less appropriate for qualifying the dynamics of diversified economies.
% 
% Altogether, the dynamics of economic activities is well explained by the proposed models, although after discounting for the total number of data points used, we find that the proposed models better characterise the dynamics of economic activities within less diversified countries with higher GDP .
% 

\begin{figure}
  \centering
  \includegraphics{figure_countries.png}
  \caption{\small \textbf{Best model fits and associated predictions for DEU, FRA and ARE, and effect of GDP on best model loglikelihoods.}
    %
    \textbf{A}-\textbf{C} Best model fits and associated predictions for DEU, FRA and ARE. Scatter dots represent the empirical data, plain lines correspond to the best model fits, and dotted lines correspond to the best model predictions. The inset plots provide the $\Delta \BIC$ for each country.
    % 
    \textbf{D} Effect of GDP on model loglikelihood. The plain line corresponds to a linear fit.
  }\label{fig:fits}
\end{figure}
\FloatBarrier

\subsection{Evidence for eco-evolutionary processes}

Applying the model selection procedure detailed in the controlled experiment, we investigate in details the effect of eco-evolutionary processes on the dynamics of the countries investigated. We find that the null model is rejected in 45 of the 77 countries, in favor of alternative models (\cref{fig:aggregate_countries}\textbf{B}).
% 
Among the alternative models, $\modalphap$, capturing positive interactions between economic activities, is the most frequently supported (\cref{fig:aggregate_countries}\textbf{B},\textbf{C}), and is given considerably more support against $\modnull$ than $\modalphan$ and $\modmu$ (\cref{fig:aggregate_countries}\textbf{A},\textbf{D}).
%
The model with spatial dispersal, $\moddelta$, is the second most supported (\cref{fig:aggregate_countries}\textbf{B},\textbf{C}) and is given the most support against $\modnull$ (\cref{fig:aggregate_countries}\textbf{A},\textbf{D}).
% 
The models capturing negative interactions, $\modalphan$, and capturing economic activity transformations, $\modmu$, are the least supported and given the least strength-of-evidence (\cref{fig:aggregate_countries}\textbf{B},\textbf{D}), but are still ranked as most supported models in some countries (\cref{fig:aggregate_countries}\textbf{A}). 

%% 
In contrast to $\modalphan$, $\moddelta$, $\modmu$ and $\modalphap$, $\moddelta$, $\modmu$, which implement complementary processes, $\modalphan$ and $\modalphap$ are structurally antagonistic, but are nonetheless equally supported in some countries (see, e.g., FIN and FRA in \cref{fig:aggregate_countries}\textbf{A}). This ambiguity highlights that the success of each model arises from their ability to capture different features of the empirical data.
% 
For instance, $\modalphan$ is the only model capturing capital decrease (see, e.g., ARE in \cref{fig:fits,figSI:fit_alphan}) or very low capital growth (see, e.g., HRV in \cref{figSI:fit_alphan}).
% 
$\modalphap$  captures well periods of exponential economic growth followed by a growth slow down, and correlation patterns in growth between activities (see, e.g., DEU and HUN in \cref{fig:fits,figSI:fit_alphap}). We find that this dynamics is associated with an increase in the self limitation coefficients $b_i$, that compensates the beneficial effect of positive interactions (\cref{fig:r_b}\textbf{B}). In contrast, the model $\modnull$ tends to overestimate the capital size for such dynamics.
%
$\moddelta$ is the only model reproducing the oscillations observed in the empirical data (see e.g. FRA in \cref{fig:fits} and BEL, GBR in \cref{figSI:fit_delta}), where oscillations arise from the term $\overline{n\hc}$ through mismatches between the local capital size and the global capital size of a given economic activity. $\moddelta$ tends to overestimate the growth rate coefficients $r_i$ against other models (\cref{fig:r_b}\textbf{A}), due to a source-sink dynamics. Indeed, the dispersal term acts as capital sink for most economic activities in most countries, where only a handful of under-developed activities, relative to global trends in the world economy, benefit from source dynamics (see, e.g., KOR in \cref{figSI:flux_delta}\textbf{C}).
% 
Characteristic trends of $\modmu$ are difficult to observe given the low number of cases where it is given statistical support. 

%%
% each model is supported against $\modnull$ in some countries, suggesting that interactions between activities, spatial dispersal, and economic activity transformations are plausible processes that may considerably affect the development of economic systems. Nonetheless,
Overall, positive interactions and spatial dispersal stand out as the most supported processes, both in term of the number of countries where they are given support, and in term of the strength-of-evidence against $\modnull$.

\begin{figure}
  \center
  \includegraphics{figure_aggregate_countriesv2_evidence_for_eco_evo.png}
  \caption{\small \textbf{Statistical support or the population dynamic models $\modalphan$, $\modalphap$, $\M_\delta$ and $\modmu$, based on Bayesian information criterion f}. 
  % 
  \textbf{A} Strength-of-evidence comparison for $\modalphan$, $\modalphap$, $\moddelta$ and $\modmu$ against $\modnull$, for countries where $\modnull$ is rejected, sorted in decreasing order. 
  %   
  \textbf{B} Number of countries where models $\modnull$, $\modalphap$, $\modalphan$, $\moddelta$, and $\modmu$ are ranked as best models. 
  %
  \textbf{C} Number of countries where models $\modalphan$, $\modalphap$, $\moddelta$ and $\modmu$ are accepted against $\M_{null}$.
  % 
  \textbf{D} Global strength-of-evidence comparison for $\modalphan$, $\modalphap$, $\moddelta$, and $\modmu$ against $\modnull$.
  %   
  \textbf{A}-\textbf{D} show that $\modalphap$ and $\moddelta$ stand out as the most supported models, both in term of number of countries where they are given support, and in term of strength-of-evidence.
  }\label{fig:aggregate_countries}
\end{figure}
\FloatBarrier


%%
% We do not find any relevant pattern between the GDP of a country and the support for a particular model. This might arise from the relatively low number of emerging economies where data is sufficient to detect eco-evolutionary processes.
% % 
% Overall, our results provide a strong evidence for the hypothesis that eco-evolutionary processes have a profound effect on economic growth.


\begin{figure}
  \center
  \includegraphics{figure_aggregate_countries_rs_bs.png}
  \caption{\small \textbf{Differences in growth rates $r_i$ and coefficients of self-limitation $b_i$ between $\modalphan$, $\modalphap$, $\moddelta$, $\modmu$, and $\modnull$}. In \textbf{A}-\textbf{B}, we aggregate the parameters $r_i$ and $b_i$ across all economic activities and all countries where models are supported. \textbf{A} shows that $\moddelta$ overestimates $r_i$ against the other models. This is because for a dispersal rate $\delta$ > 0, most economic activities experience a negative capital flux, and consequently need to grow more to compensate for the losses. \textbf{B} shows that $\modalphap$ tends to overestimate $b_i$ relative to the other models, which negative effect on growth dynamics compensates the beneficial effect of positive interactions.
   }\label{fig:r_b}
\end{figure}
\FloatBarrier


%%%%%%%%%%%%%%%%%%%%%%%%%%%%%%%%%%%%%%%%%%%%%%%%%%%%%%%%%%%%%%
%%%%%%%%%%%%%% DISCUSSION %%%%%%%%%%%%%%%%%%%%%%%%%%%%%%%%%%%%
%%%%%%%%%%%%%%%%%%%%%%%%%%%%%%%%%%%%%%%%%%%%%%%%%%%%%%%%%%%%%%
\section{Discussion}
\label{sec:discussion-econobio}
%% recap of the study
Our results indicate that dynamic models embedding eco-evolutionary processes, commonly used to characterize the dynamics of biological populations \citep{Bunin2017,Scheffer2006a,Case1990,Tilman1994b,eigen1988molecular}, can appropriately characterize the dynamics of economic activities (\cref{fig:fits}). Together with state of the art computational methods and 59 years of economic data, the models provide a framework to assess the effect of eco-evolutionary processes on the long-run development of national economies.
% 
Over the 77 countries investigated, the null model -- embedding replication and self limitation processes only -- was rejected 45 times in favor of alternative eco-evolutionary models (\cref{fig:aggregate_countries}\textbf{B}), and each of the alternative model was supported against the null model in some countries. Hence, eco-evolutionary models, that couple the growth of economic activities, better characterize their dynamics. This result suggests that economic change carries the effect of the inter-dependency between economic activities. The models capturing positive interactions between economic activities ($\modalphap$) and spatial dispersal ($\moddelta$) were overall given more strength-of-evidence, providing evidences that these processes are central in the development of economic systems.  Together, our study suggests that biologically inspired dynamic models are well suited to characterize economic dynamics. 

%%
The statistical support obtained for the model with positive interactions between economic activities and the associated parameter value estimates highlight that economic activities sustain the development of each other, supporting the hypothesis that economic diversity promotes economic growth \citep{Saviotti2020}. $\modalphap$  was the most frequently supported model (\cref{fig:aggregate_countries}\textbf{B}-\textbf{C}), outlining that positive interactions may be ubiquitous in economic systems. We also found that $\modalphap$ was associated with high self-limitation coefficients (\cref{fig:r_b}\textbf{B}). This finding highlights that the growth of economic activities is driven by positive interactions rather than by their endogenous growth. 
% 
% Evidence towards $\modalphap$ consequently implies that economic activities could hardly thrive outside of the ecosystem due to a very low endogenous growth \citep{Yukalov2012}. 
% 
Positive interactions may arise from a variety of sub-processes acting upon firms, including beneficial business interactions through supply chains \citep{Ozman2009,Saavedra2009a} and agglomeration externalities \citep{VanDerPanne2004}. Together, our results suggest that the effect of positive interactions acting upon firms scales up and at the national level and result in economic activities sustaining each other through positive feedbacks. 

%%
The dynamics of economic activities seem also to be greatly affected by spatial dispersal (\cref{fig:aggregate_countries}\textbf{B},\textbf{D}), which may originate from socio-economic processes operating at lower organizational levels and involved in the spatial diffusion of organizational routines \citep{Hodgson2004} and knowledge \citep{RogersEverettM2003DoI,Keller2004,Bahar2014a} across countries. 
%  innovation diffusion through social networks \citep{RogersEverettM2003DoI}
Candidates socio-economic processes underlying the diffusion of organizational routines are international business expansions \citep{Andersen1993,Zahra2000,Luo2007}, labor mobility \citep{Boschma2008}, and the diffusion of innovation through social networks \citep{RogersEverettM2003DoI,Keller2004}. However, we found that $\moddelta$ was less frequently supported against $\modnull$ than $\modalphap$ (\cref{fig:aggregate_countries}\textbf{C}). To explain this discrepancy, we hypothesize that transfers of knowledge and routines are blocked by barriers due to differences in cognitive, organizational, social, institutional or geographic proximity between countries \citep{Boschma2005,Caragliu2016,Bahar2014a}. This hypothesis should be investigated in future work.
% 
Overall, our results indicate that socio-economic processes involved in the transfer of knowledge and organizational routines have an important effect on economic change at the national level.

%% recap of important mechanisms highlighted
The strong discrepancies in model supports observed across countries may indicate strong idiosyncratic processes, that can relate to the countries' institutional context \citep{Acemoglu2005} or to historical contingencies \citep{Silverberg2005a}. 
% 
We found that developing economies have a distinct, less predictable dynamical behavior, as the likelihood of the best models showed a negative relationship with country GDP (\cref{fig:fits}\textbf{D} and \cref{table:gdp_loglikelihood}). In line with our findings, \citep{Cristelli2015} also reports a difference in dynamical regime between mature and developing economies, where the GDP dynamics of developing economies is found less predictable than that of rich countries. Those lower model fits indicate that singular processes may importantly influence the dynamics of developing countries.
% 
% Ranking the models with respect to their support in each country, we found a statistical advantage for $\modalphap$ and $\moddelta$, while the models $\modalphan$ and $\modmu$, accounting for negative interactions and economic activity transformations, appeared to be the least and less frequently supported.
% 
% $\moddelta$, accounting for spatial dispersal, was given more strength-of-evidence (\cref{fig:aggregate_countries}\textbf{D}), but $\modalphan$ and $\modalphap$, modelling ecological interactions, were provided statistical support in more countries (\cref{fig:aggregate_countries}\textbf{B}-\textbf{C}). 
% 
% The effect of mutualistic interactions has also been quantitatively resported on the dynamics of financial assets \citep{Gatabazi2019}, and the interdependence of firms has been reported in \citep{}. Spatial trasnfers have been reported in gravity models.
% 
Also, while we found a statistical advantage for $\modalphap$ and $\moddelta$ over the alternative models (\cref{fig:aggregate_countries}\textbf{B}-\textbf{D}), the support and strength-of-evidence for each model strongly varied across countries (\cref{fig:aggregate_countries}A), without associations with macroeconomic characteristics such as, e.g., country GDP per capita. This may be due to the noise in the data, or to the specific assumptions considered. For instance, while we assumed equal strength of positive interactions between pairs of economic sectors (\cref{eq:submodels3}), it could be that interaction strength is pair specific, implying that the likelihood estimated for the positive interaction hypothesis is underrated in countries with many economic activities. 
% 
Yet, these discrepancies could also originate from the idiosyncrasy of the country histories and characteristics.
% 
Together, our results point towards country and economic activity-specific idiosyncrasies in the nature and the role of eco-evolutionary processes in economic systems.

%% Further developments and investigation
For the sake of parsimony, the population dynamic models proposed in \cref{eq:submodels1,eq:submodels2,eq:submodels3,eq:submodels4,eq:submodels5} rely on a set of assumptions, which could be relaxed to further detail eco-evolutionary processes acting upon economic activities. 
% 
The models investigated rely on the mean field assumptions that economic activities interact with equal strengths, and proceed to spatial dispersal across countries and capital transfer between each other with equal rates.
% 
Nonetheless, similarly to biological organisms \citep{Bascompte2003}, economic entities interact through organized networks \citep{C.A.HidalgoB.Klinger,Bustos2012,Saavedra2009a,Schweitzer2009,Giuliani2007}. For instance, \cite{C.A.HidalgoB.Klinger} suggests that economic activities are related to one another through a network of relatedness, which could be integrated in $\modalphan$, $\modalphap$ and $\modmu$, weighting the interactions and transfers by the relatedness of two activities. 
% 
% We expect that constraining interactions with a known network structure might enhance the support for the process of economic activity transformations.
% 
Likewise, spatial dispersal are likely to depend on proximity metrics between countries \citep{Boschma2005,Caragliu2016,Bahar2014a}. The consideration of generalized variants of the models investigated, such as in \cref{eq:model_general}, will be required to assess the importance of economic activity relatedness and country proximity on economic dynamics. Additionally, considering the variation of the parameter values through time may be required, as the strength and directionality of interactions, spatial dispersal and economic activity transformations may temporally change throughout the development of economic activities \citep{Pistorius1997}. 
% 
Exploring more complex eco-evolutionary models will demand, however, richer time series, in order to extract the extra information needed to constrain the additional parameters.

%% General summary and conclusion
In conclusion, our results foster a biologically inspired approach to understand the mechanisms shaping the endogenous dynamics of economic systems. Using an inverse modelling approach relying on alternative population dynamic models and temporal economic data, our study demonstrates that positive interactions between economic activities and spatial dispersal considerably influence the dynamics of economic activities at the national, and may be fundamental drivers of economic change.
% 
The quantitative paradigm to study economic systems has mainly relied on models inspired from mechanics, which assume a world in equilibrium \citep{sornette2014physics}.
% 
Evolutionary biology, focusing on nonlinear dynamical processes and emergence, seems to be a more appropriate paradigm to study the fundamental forces shaping the dynamics of economic systems. It can provide quantitative models to test general hypotheses on organizational principles, and suggest new ones. We call this promising research venue \emph{econobiology}.

\section{Code availability}
The simulation code is available on request.

\section{Acknowledgements}
% The authors thank Thomas Poulet and François Duchene for helpful discussions and comments.
L.P. and V.B. were supported by the SNF grant 310030E\_205556.


% \printbibliography[heading=subbibliography]