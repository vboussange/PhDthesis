% !TEX root = ../../my-thesis.tex
\newcommand{\M}{\mathcal{M}}
\newcommand{\modnull}{\M_{null}}
\newcommand{\modalphap}{\M_{\alpha^+}}
\newcommand{\modalphan}{\M_{\alpha^-}}
\newcommand{\modmu}{\M_\mu}
\newcommand{\moddelta}{\M_\delta}
\newcommand{\hc}{^{(c)}}
\newcommand{\LL}{\mathcal{L}}
\newcommand{\BIC}{\text{BIC}}
\newcommand{\by}{\textbf{y}}

\graphicspath{{./content/chap4_econobiology/figures/}}

\chapter{Ecological interactions, spatial transfers and evolution shape the dynamics of economic activities}
\label{chap:econobiology}

by Victor Boussange$^{1,2}$,
Didier Sornette,
Heike Lischke$^{3}$,
and 
Loïc Pellissier$^{1,2}$\bigskip\\

$^1$ \small{Swiss Federal Research Institute WSL, CH-8903 Birmensdorf, Switzerland}\smallskip\\
$^2$ \small{Landscape Ecology, Institute of Terrestrial Ecosystems, Department of Environmental System Science, ETH Zürich, CH-8092 Zürich, Switzerland}\smallskip\\
$^3$ \small{Dynamic Macroecology, Land Change Science, Swiss Federal Research Institute WSL, Birmensdorf, Switzerland}\smallskip
% $^1${Email: \href{mailto:bvictor@ethz.ch}{\texttt{bvictor@ethz.ch}}}\smallskip\\
% $^2${Email: \href{mailto:loic.pellissier@usys.ethz.ch}{\texttt{loic.pellissier@usys.ethz.ch}}}\smallskip\\
% \smallskip

\clearpage

\begin{quote} %150/150 words
  \small{\textit{
    The biological processes of ecological interactions, spatial transfers and evolution have been proposed as compelling forces to explain economic growth. While this analogy has delivered useful qualitative insights, whether eco-evolutionary processes can explain economic growth has not yet been quantitatively tested.
    % 
    Building upon an analogy between economic activities and biological functional groups, we investigate whether eco-evolutionary processes can explain the dynamics of economic systems.
    %   
    We rely on an inverse modelling technique together with 59 years of economic data to estimate the statistical support of four population dynamic model embedding different implementations of eco-evolutionary processes, which we compare against a logistic growth null model.
    % 
    Modelling the temporal dynamics of 9 elemental economic activities over 71 countries, we find that the null model is rejected in 42 countries, against the alternative models. The best models capture well the development of mature economies, while developing economies display less predictable dynamics.
    % 
    Within mature economies, we find strong statistical evidence for positive and negative interactions between national economic activities, and spatial transfers across countries. These results indicate that processes akin to those displayed in biological ecosystems play a significant role in the dynamics of economies.
    % 
    Paralleling variations across different ecosystems types, the strength of such processes vary across countries, and might underlie differences in socio-political contexts, historical contingencies and national resources.
    % 
    Overall, our study provides a new quantitative, biologically inspired framework to study the forces shaping the development of economic system.
  }}
  \end{quote}
  \newpage

  \section{Introduction}

  %%%%%%%%%%%%%%%%%%%%%%%%%%%%%%%%%%%%%%%%%%%%%%%%%%%%%%%%%%
  %%%%% First paragraph economics / physics / biology %%%%%%
  %%%%%%%%%%%%%%%%%%%%%%%%%%%%%%%%%%%%%%%%%%%%%%%%%%%%%%%%%%
  The fields of evolutionary biology and economics have mutually exchanged ideas for the past two centuries \cite{Dopfer2007}, and analogies between processes in biology and economics have been increasingly used during the 20th century to explain economic development \cite{Ruth1996}.
  % 
  A pioneer was Schumpeter, who notoriously proposed that economic dynamics is driven by innovations that transform the economy -- so called periods of "creative destruction" \cite{schumpeter2017theory}-- similar to the punctuated equilibrium changes observed throughout the development of life on Earth \cite{gould1972}.
  %
  The analogy with biology was further developed within the field of evolutionary economics \cite{Hodgson2019}, largely promoted by the seminal work of \cite{nelson1985evolutionary}.
  % 
  The premise of evolutionary economics considers habits, customs, and organizational routines as "replicators", i.e. atomic units playing the role of genes and defining as a whole the cohesive identity of an economic institution \cite{Hodgson2019}.
  % 
  Under this framework, firm business strategies determine how they transform commodities and knowledge into new knowledge, new technologies or other industrial products with more added value.
  % 
  The replicators define the fitness of an economic institution within a given economic context, which grows in terms of assets and human capital and survive within an ecosystem of institutions \cite{Hodgson2002}, continuously adapting and experiencing evolutionary processes \cite{Veblen1898}.
  % 
  While this analogy has resulted in useful insights on the plausible drivers of economic growth \cite{Dopfer2007}, the proposed pathways have mostly been qualitatively investigated.
  % 
  Computational tools have recently allowed to test biological hypotheses against data to gain a quantitative understanding on the ecological and evolutionary processes shaping the dynamics of biological systems \cite{Pontarp2019,Boussange2022,Boussange2022a}. These tools could leverage the qualitative insights gained from the biology and provide a quantitative framework to investigate the forces shaping economic development.
  % \todo[inline]{econobiology discussed in this article \cite{Rosser2010}}
  
  %%%%%%%%%%%%%%%%%%%%%%%%%%%%%%%%%%%%%%%%%%%%%%%%%%%%%%%%%%
  %%%%%                 processes biology / economics %%%%%%
  %%%%%%%%%%%%%%%%%%%%%%%%%%%%%%%%%%%%%%%%%%%%%%%%%%%%%%%%%%
  Interactions between biological organisms, spatial transfers via dispersal and evolution of phenotypes represent fundamental processes that drive the dynamics of ecosystems \cite{Vellend2010}, and similar forces may shape the dynamic of economic systems.
  %
  Similarly to biological organisms, economic institutions interact in a mutualistic or competitive fashion \cite{Pistorius1997}. While biological organisms engage in negative interactions when e.g.~competing for similar resources \cite{GRIME1973}, economic institutions have negative effects on each other when relying on workforce with similar knowledge or attracting similar potential investors \cite{Wernerfelt1989}.
  % 
  Positive interactions are observed between biological organisms with e.g the accumulation of nutrients, the provision of shade, or the protection from herbivores \cite{Wernerfelt1989,Callaway2002}. In parallel, positive effects between economic institutions appear when they are connected through supply chains \cite{Ozman2009,Saavedra2009a}, with the building of a common infrastructure of innovation and the settlement of elite workers \cite{Cohendet2018}, and with knowledge spillovers \cite{Menon2015}.
  %%
  Colonization and migration processes play a major role in the development of ecosystems, where dispersal mechanisms initiate ecological succession with the settlement and growth of new species \cite{Leibold2004}. The diffusion of innovations \cite{RogersEverettM2003DoI} and international business expansions \cite{Zahra2000} are analogous mechanisms, where knowledge and organizational routines spread across space, and firms settle in foreign countries.
  %%
  Finally, evolutionary processes allow the adaptation of organisms to changing environmental conditions \cite{Bell2017}. In economic systems, economic institutions adapt to economic contexts \cite{Cordes2006} and transform into new economic institutions \cite{Freeman2002,Hodgson2004,Aldrich2008}. %An example of evolutionary processes is the innovations in the oil, gas and synthetic material industry that lead to the motorization of transport, civil economy and war, and to the mechanisation of agriculture \cite{Freeman2002}. 
  %  
  While economic interactions, spatial transfers and economic activity transformations are regularly documented in evolutionary economics, the significance of these processes on long-term economic development has been seldom quantified using economic data.%\todo{Talk about diffusion in the product space: \cite{Hidalgo2021}}

  %%%%%%%%%%%%%%%%%%%%%%%%%%%%%%%%%%%%%%%%%%%%%%%%%%%%%%%%%%
  %%%%% Literature review on quantitative models      %%%%%%
  %%%%%%%%%%%%%%%%%%%%%%%%%%%%%%%%%%%%%%%%%%%%%%%%%%%%%%%%%%
  Focusing on endogenous forces and borrowing concepts and methods from biology, a number of modelling approaches have broken the lines of standard economic models and approaches to quantitatively investigate the processes driving economic patterns \cite{Tacchella2018}.
  % 
  For instance, \cite{Saavedra2009a} investigated the effect of cooperation within firms with a model of consumer–resource interactions. \cite{Scholl2020} developed a theory of market ecology to interpret market phenomena and predict market behavior, where financial trading strategies are the analogous to biological species.
  % 
  Lotka-Volterra models were used in the prediction of technology evolution \cite{Zhang2018}, to evaluate product and firm competition \cite{Modis1997,Saavedra2014}, to understand the drivers of market share dynamics \cite{Farmer1999,Michalakelis2011,Marasco2016,Gatabazi2019} and to estimate the value of firms \cite{Cauwels56}.
  % 
  \cite{Applegate2021} investigated an ecological model of competition-colonization dynamics to understand the distribution of firm sizes, and \cite{Suweis2015} used a population dynamic model to evaluate the effect of international trade on global food security.
  % 
%   Nonetheless, the confrontation of the model predictions has remained mostly qualitative, in that patterns rather than full model outputs have been compared with observations.
  % 
  The availability of fine grained datasets of economic activities, together with dimensionality reduction techniques, have provided insights into the endogenous processes shaping economic development \cite{Mealy2019,Hidalgo2021}.
  % 
  In particular, the detailed global trade data compiled by United Nations statistical Division and cleaned by \cite{Hidalgo2021} consists in a time series of 59 years of economic activity that can be combined with inverse modelling techniques to learn about the processes influencing the long-term dynamics of economic systems.
  
  %%%%%%%%%%%%%%%%%%%%%%%%%%%%%%%%%%%%%%%%%%%%%%%%%%%%%%%%%%
  %%%%% 4th paragraph : results                       %%%%%%
  %%%%%%%%%%%%%%%%%%%%%%%%%%%%%%%%%%%%%%%%%%%%%%%%%%%%%%%%%%
  Here, we investigate whether eco-evolutionary processes can quantitatively explain economic growth using an inverse modelling technique together with 59 years of data on economic activity. Relating the temporal evolution of the capital of economic activities to the temporal evolution of the biomass of functional groups within an ecosystem, we quantitatively assess the effect of eco-evolutionary processes on the long-term development of national economies.
  % 
  Functional groups, as the aggregation of species sharing similar characteristics and having similar functions within the ecosystem, are the required elemental units to model ecosystem dynamics. Analogously, as the aggregation of firms with similar output nature \cite{Applegate2021}, we consider 9 economic activities to form the elemental units of economic systems, and  conduct our investigation by modelling their temporal dynamics.
  % 
  We consider a null model implementing the fundamental processes of self-replication and self-limitation ($\modnull$), where no couplings between economic activities are captured, which we contrast to alternative  dynamic models that capture ecological interactions ($\modalphan,\modalphap$), spatial transfers ($\moddelta$) and economic activity transformations ($\modmu$).
  % 
  We use the mini-batching machine learning technique of \cite{Boussange2022a} to estimate the maximum likelihood of each model for 100 countries, and use a model selection technique to evaluate the statistical support of each alternative model, based on a data set of national exports from 1962 to 2020, taken as a proxy for the temporal development of the capital of economic activities. 
  %
  We first show that the machine learning method together with the model selection procedure can provide support for the generating processes in a controlled experiment.
  %   
  Further applying the model selection procedure to the empirical data, we find evidences for eco-evolutionary processes, and observe differences in model support across countries, which may reflect idiosyncrasies of economic development. 
  %%
  By adopting a biological perspective on economic dynamics, our study sets the basis for an understanding of the endogenous forces determining economic growth, and proposes a complementary perspective on the drivers of economic dynamics.
\begin{comment}
    As in previous work \cite{Michalakelis2011}, 
    \cite{Brummitt2020}: analysis of time series to understand patterns of growth. For sure to be cited. Our work goes in this direction, and further provides other distinction, i.e. evolution is a distinguishable factor.
\end{comment}

%%%%%%%%%%%%%%%%%%%%%%%%%%%%%%%%%%%%%%%%%%%%%%%%%%%%%%%%%%%%%%
%%%%%%%%%%%%%% METHODS    %%%%%%%%%%%%%%%%%%%%%%%%%%%%%%%%%%%%
%%%%%%%%%%%%%%%%%%%%%%%%%%%%%%%%%%%%%%%%%%%%%%%%%%%%%%%%%%%%%%   

\section{Methods}\label{sec:methods}

\subsection{Eco-evolutionary model to characterise the dynamics of economic activities}

We derive a general population dynamic model where the dynamics of national economic activities is driven by the fundamental processes of self-replication, self-limitation, interactions with other activities, spatial transfers between countries, and activity transformations within the country. We present the general model in the following, which we further split into alternative sub-models to test the support of each process.

In the general population dynamic model that includes all the processes investigated, we consider that the rate of change of the size of an economic activity $i$ in the country $c$, denoted as $n_i\hc $, follows
%; see \cite{Nordbotten} for a variant model where populations are structured over a continuous space.

\begin{equation}\label{eq:model_general}
    \tfrac{d}{dt} n_i\hc(t)  = r_i\hc  n_i\hc(t) \left(1 -   b_i\hc  n_i\hc(t)  + \sum_{j \neq i}^{N\hc} \alpha\hc _{i,j} n_j\hc(t) \right) + \delta\hc_{i} \sum_l^M \left( n_i^{(l)}(t) - n_i\hc(t)  \right) + \sum_j^{N\hc} \mu\hc _{j,i}\left( n_j\hc(t)  - n_i\hc(t)  \right),
\end{equation}
where $N\hc$ is the number of economic activities considered in country $c$ and $M$ is the number of countries considered.
% 
In \cref{eq:model_general}, the first summand corresponds to a Lotka Volterra model for ecological communities (see, e.g., \cite{Bunin2017,Scheffer2006a,Case1990}), where  $r_i\hc $ is the growth rate, involved in the self-replication, and $b_i\hc $ is the self-interaction coefficient, involved in the self-limitation. $b_i\hc $ can also be interpreted as the inverse of the country's carrying capacity for the activity. %\cite{Grimme}
%
The individual dynamics of economic activities are altered by interactions, where $\alpha_{i,j}\hc $ captures the interaction between activity $ i $ and activity $ j $ at location $c$. The interaction between $i$ and $j$ is mutualistic if $\alpha_{i,j}\hc  > 0$, and competitive when $\alpha_{i,j}\hc  < 0$, altering positively or negatively the rate of change of the activity $i$ by the presence of other activities.
%
The second summand accounts for spatial transfers across locations (see, e.g., \cite{Tilman1994b}), where $\delta^{(l,c)}_{i}$ is the rate of spatial transfer for activity $i$ between locations $l$ and $c$.
% 
The last summand corresponds to economic activity transformations within country $c$, where $\mu\hc _{j,i}$ is the rate of transfer dictating how fast activity $j$ transforms into activity $i$. In the sense of the quasi-species model \cite{eigen1988molecular}, this term can be interpreted as accounting for evolutionary processes, and can contribute to the development of an activity in the presence of other activities.
%
A graphical representation of the model is given in \cref{fig:model}.

%%
\Cref{eq:model_general} involves a high number of independent parameters to fit (i.e., $N\hc(4+N\hc)$ parameters for each country), which can be reduced for the sake of parsimony under mean field assumptions. Specifically, we assume that all economic activities interact similarly, so that $\alpha_{i,j}\hc  = \alpha\hc $. We assume that activity transformations are symmetric and occur at similar rates for all activities, so that $\mu_{i,j}\hc  = \mu\hc $, and we assume that spatial transfers occur at similar rates for all activities, so that $\delta\hc _{i} = \delta^{(c)}$. These mean field assumptions reduce the number of parameters to $2N+3$. 
% 
Overall, the parameters $r_i\hc$ and $b_i\hc$ are assumed activity- and country-dependent, and $\alpha\hc$, $\mu\hc$, and $\delta\hc$ are assumed country-dependent. The value of these parameters is assumed to be determined by the activity's characteristics and the country's socio-political system (comprising e.g.~the taxation regime, the system of innovation, the legal system, the intellectual property right and the socio-cultural background of the society) and the country's resources (comprising e.g.~labour force, knowledge capital, agricultural resources, mineral resources and energy resources), but does not change through time.

%%
To investigate how the data supports each of the process embedded in \cref{eq:model_general}, we further decompose the model into five different sub-models $\M_{null}$, $\modalphap$, $\modalphan$, $\moddelta$, and $\modmu$ as follows

  \begin{align}
  \label{eq:submodels1}
    \M_{null}&:  &&\tfrac{d}{dt} n_i\hc(t)  = r_i\hc  n_i\hc(t) (1 -  b_i\hc n_i\hc (t) ) \\
  \label{eq:submodels2}
    \modalphap&: && \tfrac{d}{dt} n_i\hc(t)  = r_i\hc  n_i\hc(t) \left(1 -  b_i\hc n_i\hc(t) + \alpha\hc  \sum_{j \neq i} n_j\hc(t) \right), \quad &&\alpha > 0\\
      \label{eq:submodels3}
    \modalphan&: && \tfrac{d}{dt} n_i\hc(t)  = r_i\hc  n_i\hc(t) \left(1 -  b_i\hc  n_i\hc(t) + \alpha\hc  \sum_{j \neq i} n_j\hc(t) \right), \quad &&\alpha < 0\\
      \label{eq:submodels4}
    \moddelta&: && \tfrac{d}{dt} n_i\hc(t)  = r_i\hc  n_i\hc(t) (1 - b_i\hc  n_i\hc(t) ) + (\delta\hc )^2 \left(\overline{n_i\hc}(t) - n_i\hc(t) \right) \\
      \label{eq:submodels5}
    \M_{\mu}&: && \tfrac{d}{dt} n_i\hc(t)  = r_i\hc  n_i\hc(t) (1 -  b_i\hc n_i\hc(t) ) + (\mu\hc )^2 \sum_j^{N\hc} \left( n_j\hc  (t) - n_i\hc(t)  \right)
  \end{align}

% Mathematical properties of equivalent model have theoretically been investigated in 
% \begin{itemize}
%     \item 
%         \cite{Rohr2014}: structural stability = dynamically stable and feasible equilibrium. It is shown that an interaction matrix that is Lyapunov–diagonally stable grants the global stability of any potential feasible equilibrium.
%     \item 
%         ??
% \end{itemize}
where 
\begin{equation}
      \overline{n_i\hc } = \sum_{l \neq c}^M n_{i}^{(l)}
\end{equation}
accounts for the capital of activity $i$ at the global level, excluding country $c$. 
% 
The simplest model $\modnull$ embeds self-replication and self-limitation without any further coupling force acting upon economic activities, and is considered as the null model.
% 
The alternative models $\modalphap$, $\modalphan$, $\moddelta$, and $\modmu$ capture self-replication and self-limitation together with one of the additional eco-evolutionary process investigated.
% 
Because we find inconsistent maximum likelihood estimations in the controlled experiment detailed in the \nameref{sec:results} for models incorporating combinations of the processes in \cref{eq:submodels2,eq:submodels3,eq:submodels4,eq:submodels5}, we do not test more complex models.
% 
In the following, we designate by $\M(t,\theta\hc )$ the vector of economic activity capitals predicted by model $\M$ at time $t$ for country $c$ with the parameter vector $\theta\hc$, that includes the growth rates $r_i\hc $, the self limitation rates $b_i\hc$, the specific model parameters $\alpha\hc$, $\mu\hc$, $\delta\hc$, and the initial condition vector $n\hc (t_0)$ for country $c$.

% The model is structurally identifiable (i.e parameters can be infered from time series) \cite{Remien2021}. 


\begin{figure}[ht]
  % \includegraphics[width=0.8\textwidth]{img/conceptual.png}
  \includegraphics[width=0.8\textwidth]{conceptual/model_tikz.pdf}
  \centering
  \caption{\small
      \textbf{General population dynamic model used to characterise the dynamics of economic activities, capturing ecological interactions, spatial transfers and economic activity transformations.}
      % 
      For each country $c$, each one of the 9 economic activities considered is assumed to be driven by replication and self limitation processes, characterised respectively by a growth rate $r_i\hc$ and a self-limitation coefficient $b_i\hc$, which values are assumed to be country-dependent and determined by the socio-political context and resources available (in terms of human capital, knowledge capital and natural resources). 
      % 
      In addition, interactions between activities within each country are determined by $\alpha\hc$ and can be mutualistic ($\alpha\hc > 0$) or competitive ($\alpha\hc< 0$).
      % 
      Due to spatial transfers, the capital of an economic sector can experience positive (resp. negative) fluxes of capital when it is lower (resp. higher) than the average global capital for the economic sector at time $t$, arising at rate $\delta\hc$.
      % 
      Due to economic activity transformations, the capital of an economic sector can experience positive (resp. negative) fluxes of capital when the capital it is lower (resp. higher) than the capital of other economic sectors within the country at time $t$, arising at rate $\mu\hc$.
      % 
      The parameters $r_i\hc$, $b_i\hc$, $\alpha\hc$, $\delta\hc$ and $\mu\hc$ that best fit the empirical data are estimated in each country for each sub-model in \cref{eq:submodels1,eq:submodels2,eq:submodels3,eq:submodels4,eq:submodels5}, to investigate whether the proposed processes have a significant effect on the temporal evolution of the capital of economic activities.
      }\label{fig:model}
\end{figure}
\FloatBarrier

    
\subsection{Empirical data and model likelihood}

We consider time series of global trade data as a proxy for the evolution of the capital of economic activities through time. Export data is synthetic of the capital of an economic activity within a country because the more it is exporting, the more it is competing in the global market, and therefore the larger it is in terms of capital \cite{Tacchella2018}. We specifically use the database compiled by the United Nations Statistical Division COMTRADE, categorized in the Standard International Trade Classification (SITC, revision 2) at the 1 digit level, covering 11 categories of economic activities in 249 countries from 1962 to 2020. Because of their low quality \cite{Hidalgo2021}, we discard the "Unspecified" and "Services" category, resulting in 9 economic categories to model (see \cref{fig:fits} for details). As shown in the \nameref{sec:results}, this large number of data points contains the information necessary to recover the plausible generating processes.
%%
Export values for activity $i$ in country $c$, denoted by $X_{i}\hc (t)$, are discounted by the national population at time $t$, denoted by $P\hc (t)$, in order to compare the economic variables across time and across countries. This leads to expressing the observation data for the capital of activity $i$ in country $c$ as 
% 
$ y_i\hc (t) = X_{i}\hc (t) / P\hc (t) $.
% 
%although RCA is usually further scaled by the ensemble export product within the country as the end goal of complexity economics is not to characterise growth but specialisation of an economy.
%%
In each country, we only consider economic activities which have sustained a significant size relative to the world trends for at least 4 years (revealed comparative advantage $>1$, see \cite{Hidalgo2021}).
% 
We assume that the observation data is contaminated by a white noise with log-normal distribution $\epsilon$ with zero mean and a variance-covariance matrix $\Sigma = \sigma^2 I$ where $\sigma$ is the noise level, which is a reasonable error model for population dynamics \cite{Schartau2017}, so that
% 
$y\hc _i(t)  = n\hc _i(t) \exp(\epsilon_i\hc (t))$.
% 
This leads to expressing the likelihood of model $\M$ in country $c$, denoted by $\LL(\theta\hc_\M | {\by\hc }, \M)$, as

\begin{equation}\label{eq:likelihood_lognormal}
  \LL(\theta\hc_\M | {\by\hc }, \M) = \prod_{j=1}^{T\hc} \frac{1}{\sqrt{(2\pi)^{N\hc}|\Sigma|} y\hc(t_j)} \exp \left(-\frac{1}{2} \Big[ \tilde{d}\hc(t_j)\Big]^{T\hc} \Sigma^{-1} \Big[ \tilde{d}\hc(t_j)\Big] \right)
\end{equation}
% 
where $\theta\hc_\M$ refers to the parameters and the initial conditions for model $\M$, $\by\hc = (y\hc(t_1), \dots, y\hc(t_T))$  designate the $T\hc$ time points of economic data available for country $c$, $ y\hc(t) = (y_1(t), \dots, y\hc_N(t_T))$ is the vector of economic activity capital at time $t$, and $\tilde{d}\hc(t_j) = \ln(y\hc(t_j)) - \ln(\M(t_j,\theta\hc_\M))$. 
% 
We detail in the next section how do we obtain the maximum likelihood estimate $\hat\theta\hc_\M$ that maximises \cref{eq:likelihood_lognormal} given $\by\hc$ for each model $\M$ and country $c$, further used for selecting the most probable model given the data \cite{Burnham2002}. In the following, we drop the indices $\hc$ for clarification.

\subsection{Maximum likelihood estimation}\label{parameter-inference}

In order to obtain the maximum likelihood of each model given the data, we employ the machine learning framework detailed \cite{Boussange2022a}, which is based on a variational optimisation method and a learning strategy coined "mini-batching". The likelihood of model $\M_i$ is maximised by training $\M_i$ against mini-batches of data comprising only $K < T\hc$ data points from the full time series, where both the parameters and initial conditions are estimated.
% 
The mini-batching strategy ensures the convergence of the optimization towards the maximum likelihood estimate, provided that the choice of $K$ is appropriate given the data and the model investigated \cite{Boussange2022a}. 
% 
A large $K$ might induce convergence towards a local minimum, while a low $K$ might flatten the likelihood landscape, where all models would be provided equal support. 
% 
In the controlled experiments in \cref{sec:synthetic}, we find that $K=20$ (models are trained against batches of 20 years) ensures a good convergence, while reasonably discriminating the models (\cref{fig:synthetic_test_all_AIC}). We therefore use $K=20$ throughout all the experiments detailed in the manuscript, and discard countries where data is available for less than $K = 20$ years.
% 
The library \textbf{MiniBatchInference.jl} \cite{Boussange2022a} is employed for the numerical implementation, with the gradient descent optimisers ADAM \cite{Kingma2014} during the first 800 epochs to converge in the basin of attraction of the of the maximum likelihood estimate, which we substitute for the Broyden–Fletcher–Goldfarb–Shanno optimizer (BFGS) \cite{fletcher2013practical} for the final 800 training epochs to ensure faster and more accurate convergence.
% 
As a cross checking procedure, for each country and each model we perform 5 optimization runs with different initial parameters, where $r_i$, $b_i$, $\alpha$, $\mu$, $\delta$ are drawn from the uniform random distribution $\mathcal{U}_{[0.05, 0.15]}$, $\mathcal{U}_{[0.5,1.5]}$, $\mathcal{U}_{[0.5,1.5]}$, $\mathcal{U}_{[0.0005, 0.0015]}$ and $\mathcal{U}_{[0.0005, 0.0015]}$ respectively.
% 
We then take the best run among the 5 optimization runs, making sure that the likelihood estimates of each run are similar in value.
% 
% We used a gradient descent optimisation method to minimise $L(\theta\hc )$. We used the library \textbf{MiniBatchInference.jl} to train the ODE parameters $\theta^{(c)_\M}$ with the gradient descent optimiser ADAM and BFGS. Both the parameters and the intitial conditions for each activities were fitted.

% \todo[inline]{how structural change that leads to increasing output variety was gradually perceived by economists and eventually incorporated into models of economic growth: \cite{Saviotti2020}}    
%%%%%%%%%%%%%%%%%%%%%%%%%%%%%%%%%%%%%%%%%%%%%%%%%%%%%%%%%%%%%%
%%%%%%%%%%%%%% RESULTS    %%%%%%%%%%%%%%%%%%%%%%%%%%%%%%%%%%%%
%%%%%%%%%%%%%%%%%%%%%%%%%%%%%%%%%%%%%%%%%%%%%%%%%%%%%%%%%%%%%%

\section{Results}
\label{sec:results}
\subsection{Validation with synthetic data}\label{sec:synthetic}
We first investigate whether the inverse modelling technique of \cite{Boussange2022a} together with the proposed sub-models in \cref{eq:submodels1,eq:submodels2,eq:submodels3,eq:submodels4,eq:submodels5} can detect imprints of eco-evolutionary processes in a controlled experiment, using synthetic datasets.
%
We proceed by generating multiple observation datasets from the models $\modalphan$, $\modalphap$, $\M_\delta$ and $\modmu$ with realistic $r_i$ and $b_i$ parameters and with different values for the parameters $\alpha$, $\mu$, and $\delta$ and with different values for the noise level $\sigma$. We consider $\modnull$, $\modalphan$, $\modalphap$, $\M_\delta$ and $\modmu$ as equally plausible candidate models for each generated datasets, and apply the maximum likelihood estimation method detailed in the \nameref{sec:methods} to obtain the maximum likelihood $\LL(\hat \theta, \M | \by)$ of each model for each generated dataset. We then use the Bayesian information criterion (BIC) to select the model with the strongest strength-of-evidence in relation to the data \cite{Mangan2017}, calculating the BIC for model $\M_i$ as $\BIC_{\M_i} = -2 \ln(\LL(\hat \theta, \M_i | \by)) + k_{\M} \ln(N T ) $
% 
where $N$ is the number of activities in country $c$, $T$ is the number of time points in the time series considered ($N T$ therefore being the number of data points for country $c$) and $k_{\M}$ is the number of parameters in the model $\M$ for country $c$.
% 
The BIC ranks the most probable models by penalizing complexity to balance information loss and parsimony, where candidate models with the lowest scores are ranked as the most likely \cite{Mangan2017}.
% 
We consider the relative BIC score $\Delta \BIC_{\M_i}$, which allows a strength-of-evidence comparison across models and is calculated as 
%
$\Delta \BIC_{\M_i} = \BIC_{\M_i} - \min_{j} \BIC_{\M_j}$. 
%
We expect the $\Delta \BIC$ scores to only provide support for the true generating model when the process considered has a significant effect on the observed dynamics.
%%
Under realistic observational noise ($\sigma = 0.2$), we find overall strong support for the true models when the values of the parameters $\alpha$, $\mu$ and $\delta$ exceed a certain threshold (when $\M_i$ is the true generating model, $\Delta\BIC_{\M_{i}} = 0$ and $\Delta\BIC_{\M_{j,j\neq i}} > 10$, see \cref{fig:synthetic_test_all_AIC}\textbf{A-C}). 
% 
% In particular, our results indicate that when $\Delta \BIC_{\M_i} = 0 $ and when for all the other models $\M_j$, $j\neq i$, we have that $\Delta\BIC_{\M_j} > 10$, then $\M_i$ is systematically the true model.
% 
% This is in accordance with classical model selection criterion rules \cite{Burgnam}, where it is assumed that a difference in $\BIC$ greater than 10 between two models guarantees support for the model with the lower $\BIC$.
% 
Moreover, the more extreme the value of $\alpha$, $\mu$, and $\delta$, the more support is given to the true generating model.
% 
In contrast, when the imprint of the underlying process on the dynamics is not sufficiently strong, the null model $\modnull$ is given the most strength of evidence ($\Delta\BIC_{\modnull} = 0)$.
% 
% On the other hand, $\Delta \BIC_{\modnull} < 10$ indicates that the imprint of the underlying processes on the dynamics is not strong enough, and in this case, the parsimony principle should favor the least complex model, the null model $\modnull$.
% 
These results hold for varying noise levels (see \cref{figSI:synthetic_test_all_AIC} for $\sigma = 0.3$), indicating that the models together with the maximum likelihood estimation method are well adapted to investigate the influence of eco-evolutionary processes with the dataset considered.
%
Based upon this experiment and classical model selection criterion \cite{Burnham2002}, we decide to accept the hypothesis that the null model is the best model given the data when $\Delta \BIC_{\modnull} \leq 10$. In addition, when $\BIC_{\M_i} - \BIC_{\modnull} < -10$ we conclude that $\M_i$ is supported against $\modnull$, and we conclude that $\M_i$ is the best model given the data if $\Delta \BIC_{\M_i} = 0 $ and if for all other model $\M_{j,j \neq i}$, we have that $\Delta \BIC_{\M_{j,j \neq i}} > 10$.
%%
Overall, the proposed eco-evolutionary model together with the maximum likelihood estimation leads to high discrimination ability when combined with the BIC-based model selection procedure, and can provide strength-of-evidence for eco-evolutionary processes that may shape the dynamics of economic systems.

\begin{figure}
  \centering
  \includegraphics{synthetic_test_all_AIC.png}
  \caption{\small
  \textbf{Validation of the model selection procedure.} In \textbf{A} and $\textbf{B}$ we generate data with models $\modalphan$ and $\modalphap$ with varying $\alpha$, in \textbf{B} we generate data with model $\moddelta$ with varying $\delta$ and in \textbf{C} we generate data with model $\modmu$ with varying $\mu$. We then contaminate the generated data with realistic noise ($\sigma = 0.2$), and seek for the maximum likelihood of all models $\modnull$, $\modalphan$, $\modalphap$, $\moddelta$ and $\modmu$. The maximum likelihood of each model is then used to compute difference in Bayesian Information Criterion for each model, $\Delta \BIC$, which provides strength-of-evidence for each model in relation to the data. 
  %   
  \textbf{A}-\textbf{D} show that the model selection procedure is valid, since only the true models are placed in the "Strong support" category when no other model is given support (equivalently, when all other model are placed in the "No support" category).}
  \label{fig:synthetic_test_all_AIC}
\end{figure}
\FloatBarrier

\subsection{Discrepancies in dynamical regime across countries}

We apply the maximum likelihood estimation method detailed in the \nameref{sec:methods} to obtain the maximum likelihood of each model for the world top 100 countries with the highest GDP per capita, as of 2020. We first investigate the quality of best fit, and further detail the results for each model in the following section. %as it is assumed that the corresponding data is less fluctuating \cite{Tacchella2018}.
% 
Over the 100 countries investigated, only 79 countries have sufficiently long time series to be included in the analyses (countries with at least 20 years of data, see \nameref{sec:methods}), while 71 countries show consistent maximum likelihood estimates (best model coefficient of determination $R^2$ > 0).
% 
Among the 71 countries, we find that the best models have a good fit to the data, and capture well the long term growth of economic activities (median explained variance $R^2 = 0.92$, see \cref{fig:fits} and \cref{figSI:fit_alphan,figSI:fit_alphap,figSI:fit_delta,figSI:fit_mu} for graphical illustrations of the fit).
% 
While this demonstrates the relevance of the proposed models, we observe discrepancies across countries in how well the best models perform (standard deviation of explained variance std$(R^2)= 0.07$, see \cref{fig:fits}\textbf{D} for a graphical illustration of the variance in terms of the model loglikelihoods).
% 
% the loglikelihood of a country is a very good predictor of the logarithm of its current GDP (\cref{fig:fits}\textbf{D} and \cref{table:gdp_loglikelihood}, $R^2 = 0.65$), and explain this strong association by two reasons.
% 
This discrepancy is explained by the number of data points available in each country, as it differs among countries and since more data points yield a higher model loglikelihood ($\beta = 0.915 \pm 0.052$, $p<0.001$, see \cref{table:gdp_loglikelihood}).
% 
We additionally observe a positive effect of the GDP on the residuals ($\beta = 0.834 \pm 0.189$, $p<0.001$, see \cref{table:gdp_loglikelihood} and \cref{fig:fits}\textbf{D}), where rich economies have a more deterministic behavior than poorer ones. The residuals are further negatively related to the number of economic activities ($\beta = -0.288 \pm 0.071$, $p<0.001$, see \cref{table:gdp_loglikelihood}), indicating the models investigated might be less appropriate for qualifying the dynamics of diversified economies.
% 
% Altogether, the dynamics of economic activities is well explained by the proposed models, although after discounting for the total number of data points used, we find that the proposed models better characterise the dynamics of economic activities within less diversified countries with higher GDP .
% 

\begin{figure}
  \centering
  \includegraphics{figure_countries.png}
  \caption{\small \textbf{Best model fits and associated predictions for DEU, FRA and ARE, and effect of GDP on best model loglikelihoods.}
    %
    \textbf{A}-\textbf{C} Best model fits and associated predictions for DEU, FRA and ARE. Scatter dots represent the empirical data, plain lines correspond to the best model fits, and dotted lines correspond to the best model predictions. The inset plots provide the $\Delta \BIC$ for each country.
    % 
    \textbf{D} Effect of GDP on model loglikelihood. The plain line corresponds to a linear fit.
  }\label{fig:fits}
\end{figure}
\FloatBarrier

\subsection{Evidence for eco-evolutionary processes}

Applying the model selection procedure detailed in the controlled experiment, we investigate the effect of eco-evolutionary processes on the dynamics of the countries investigated. We find that the null model is rejected in 42 of the 71 countries, in favor of alternative models (\cref{fig:aggregate_countries}\textbf{B}).
% 
The models $\modalphap$ and $\modalphan$, both capturing interactions across activities, are altogether the most frequently supported (\cref{fig:aggregate_countries}\textbf{C}),
%
while the model with spatial transfers, $\moddelta$, is given the most support against $\modnull$ (\cref{fig:aggregate_countries}\textbf{D}).
% 
In contrast, the model capturing economic activity transformations, $\modmu$, is supported against $\modnull$ in the least number of countries (\cref{fig:aggregate_countries}\textbf{C}).
% 
While $\modalphan$, $\moddelta$, $\modmu$ and $\modalphap$, $\moddelta$, $\modmu$ implement complementary processes, $\modalphan$ and $\modalphap$ are structurally antagonistic, but are equally supported in some countries (see, e.g., FIN and FRA in \cref{fig:aggregate_countries}\textbf{A}). This ambiguity highlights that the success of each model arises from their ability to capture different features of the empirical data.
% 
For instance, $\modalphan$ is the only model capturing capital decrease (see, e.g., ARE in \cref{fig:fits,figSI:fit_alphan}) or very low capital growth (see, e.g., HRV in \cref{figSI:fit_alphan}).
% 
$\modalphap$  captures well periods of exponential economic growth followed by a growth slow down, and correlation patterns in growth between activities (see, e.g., DEU and HUN in \cref{fig:fits,figSI:fit_alphap}). We find that this dynamics is associated with an increase in the self limitation coefficients $b_i$, that compensates the beneficial effect of positive interactions (\cref{fig:r_b}\textbf{B}). In contrast, the model $\modnull$ tends to overestimates the capital size for such dynamics.
%
$\moddelta$ is the only model reproducing the oscillations observed in the empirical data (see e.g. FRA in \cref{fig:fits} and BEL, GBR in \cref{figSI:fit_delta}), where oscillations arise from the term $\overline{n\hc}$ through mismatches between the local capital size and the global capital size of a given economic activity. $\moddelta$ tends to overestimate the growth rate coefficients $r_i$ against other models (\cref{fig:r_b}\textbf{A}), due to a source-sink dynamics. The dispersal term act as capital sink for most economic activities in most countries, where only a handful of under-developed activities, relative to global trends in the world economy, are dependent on global economic trends and benefit from source dynamics (see, e.g., KOR in \cref{figSI:flux_delta}\textbf{C}).
% 
Characteristic trends of $\modmu$ are difficult to observe given the low number of cases where it is given statistical support. 
% 
Overall, model supports and fits to data suggest that interactions between activities, spatial transfers, and economic activity transformations play a major role in the development of economic systems.

\begin{figure}
  \center
  \includegraphics{figure_aggregate_countriesv2_evidence_for_eco_evo.png}
  \caption{\small \textbf{Statistical support based on Bayesian Information Criterion for the population dynamic models $\modalphan$, $\modalphap$, $\M_\delta$ and $\modmu$, implementing eco-evolutionary processes}. 
  % 
  \textbf{A} Strength of evidence for $\modalphan$, $\modalphap$, $\moddelta$ and $\modmu$, for countries where $\modnull$ is rejected. 
  %   
  \textbf{B} Number of countries where models $\modnull$, $\modalphap$, $\modalphan$, $\moddelta$, and $\modmu$ are ranked as best models. 
  %
  \textbf{C} Number of countries where models $\modalphan$, $\modalphap$, $\moddelta$ and $\modmu$ are accepted against $\M_{null}$.
  % 
  \textbf{D} Strength of support for $\modalphan$, $\modalphap$, $\moddelta$, and $\modmu$ against $\modnull$.
  }\label{fig:aggregate_countries}
\end{figure}
\FloatBarrier


%%
% We do not find any relevant pattern between the GDP of a country and the support for a particular model. This might arise from the relatively low number of emerging economies where data is sufficient to detect eco-evolutionary processes.
% % 
% Overall, our results provide a strong evidence for the hypothesis that eco-evolutionary processes have a profound effect on economic growth.


\begin{figure}
  \center
  \includegraphics{figure_aggregate_countries_rs_bs.png}
  \caption{\small \textbf{Differences in growth rates $r_i$ and coefficients of self-limitation $b_i$ between $\modalphan$, $\modalphap$, $\moddelta$, $\modmu$, and $\modnull$,}. In \textbf{A}-\textbf{B}, we aggregate the parameters $r_i$ and $b_i$ across all countries and all economic activities. \textbf{A} shows that $\moddelta$ overestimates $r_i$ against the other models. This is because for dispersal rate $\delta$ > 0, most economic activities experience a negative capital flux, and consequently need to grow more to compensate for the losses. \textbf{B} shows that $\modalphap$ tends to overestimates $b_i$ relative to the other models, which negative effect on growth dynamics compensates the beneficial effect of positive interactions.
   }\label{fig:r_b}
\end{figure}
\FloatBarrier


%%%%%%%%%%%%%%%%%%%%%%%%%%%%%%%%%%%%%%%%%%%%%%%%%%%%%%%%%%%%%%
%%%%%%%%%%%%%% DISCUSSION %%%%%%%%%%%%%%%%%%%%%%%%%%%%%%%%%%%%
%%%%%%%%%%%%%%%%%%%%%%%%%%%%%%%%%%%%%%%%%%%%%%%%%%%%%%%%%%%%%%
\section{Discussion and conclusion}

%% recap of the study
Our results provided a quantitative support for the hypothesis that interactions between economic activities, spatial transfers, and economic activity transformations play an important role in the development of economic systems. 
% 
The population dynamic models in \cref{eq:submodels1,eq:submodels2,eq:submodels3,eq:submodels4,eq:submodels5} (\cref{fig:model}), commonly used to characterise the dynamics of biological populations \cite{Bunin2017,Scheffer2006a,Case1990,Tilman1994b,eigen1988molecular}, characterised well the dynamics of economic activities (\cref{fig:fits}). Together with state of the art computational methods and 59 years of economic data, they provided a framework to assess the effect of eco-evolutionary processes on the long-term development of national economies.
% 
Over the 71 countries investigated, the null model -- embedding replication and self limitation processes only -- was rejected 42 times in favor of alternative eco-evolutionary models (\cref{fig:aggregate_countries}\textbf{B}). The alternative models couple the individual dynamics of economic activities through different processes proposed in the evolutionary economics literature \cite{Hodgson2019}. This results highlights that long-term economic development carries the effect of the inter-dependency between economic activities, and that biological models are well suited to characterise this coupling. %, as all the eco-evolutionary processes investigated couple the individual dynamics of economic activities.

%% recap of important mechanisms highlighted
We found that developing economies have a distinct, less predictable dynamical behavior, as the strength-of-evidence of the best model (loglikelihood) showed a negative relationship with country GDP (\cref{fig:fits}\textbf{D}). Interestingly, \cite{Cristelli2015} also reports a difference in dynamical regime between mature and developing economies, where the GDP dynamics of developing economies is found less predictable than that of rich countries. These poorer model fits are important sources of information \cite{Curtsdotter2019}, and indicate that singular processes may importantly influence the dynamics of developing countries.
% 
Ranking the models in each country, we did not find any statistical advantage between $\modalphan$, $\modalphap$, and $\moddelta$, while the model $\modmu$, accounting for economic activity transformations, appeared to be the least and less frequently supported.
% 
% $\moddelta$, accounting for spatial transfers, was given more strength-of-evidence (\cref{fig:aggregate_countries}\textbf{D}), but $\modalphan$ and $\modalphap$, modelling ecological interactions, were provided statistical support in more countries (\cref{fig:aggregate_countries}\textbf{B}-\textbf{C}). 
% 
% The effect of mutualistic interactions has also been quantitatively resported on the dynamics of financial assets \cite{Gatabazi2019}, and the interdependence of firms has been reported in \cite{}. Spatial trasnfers have been reported in gravity models.
% 
Neither did we find a pattern between the GDP of a country and the support for a particular model. This idiosyncray may be due to the relatively low number of countries where eco-evolutionary processes could be detected, to the oversimplicity of the model investigated, to the quality of the empirical data, or may be inherent to the idiosyncracy of country history and characteristics.
% 
Along the latter hypothesis, our results indicate that the strength of ecological interactions, spatial transfers and economic activity transformations and their effects on long-term economic development may be highly dependent on socio-political contexts, historical contingencies and national resources.

%% Interpretation of the different models
Reproducing specific features of economic growth, each model provides insights on credible mechanisms driving economic development.
% 
$\modalphap$ attributes the growth of economic activities to positive interactions rather than to their endogenous growth, since its dynamics is associated with high self-limitation coefficients (\cref{fig:r_b}\textbf{B}). This suggests that economic activities sustain each other through symbiosis, possibly because economic activities are involved in interdependent supply chains \cite{Ozman2009,Saavedra2009a}. Evidence towards $\modalphap$ consequently implies that economic activities could hardly thrive outside of the ecosystem due to a very low endogenous growth \cite{Yukalov2012}.
% 
$\modalphan$ captures the deleterious effect that can have some economic activities on others \cite{Wernerfelt1989}, where the most competing activity is to outcompete any other activity by tapping the entirety of the resources available (see e.g. the vegetable oil production predicted to overtake the fuel production in ARE in \cref{fig:fits}).
% 
Strong strength-of-evidence in favor of $\moddelta$ (\cref{fig:aggregate_countries}\textbf{D}) highlights that spatial transfers may greatly affect economic development. This support suggests that analogously to biological populations, economic activities spread across geographical space, possibly to increase profitability and reach new markets \cite{Luo2007}.
% 
A closer look at the effect of fluxes due to spatial tranfers on growth dynamics shows that, relative to the capital of economic activities, positive fluxes decrease over time in some countries (see, e.g., KOR in \cref{figSI:flux_delta}\textbf{B}). This pattern resembles a colonization process where net positive fluxes of biomass first contribute to the growth dynamics, while after some time, endogenous growth becomes the most important mechanism driving the dynamics \cite{Leibold2004}. 
%
Lastly, the support for $\modmu$ indicates that predominant economic activities significantly transfer capital to smaller activities. This phenomenon evokes the diffusion process proposed in \cite{C.A.HidalgoB.Klinger} and is analogous to the processes of ecological succession \cite{Odum1969}, where new activities appear and develop thanks to spillovers an knowledge diffusion from other activities \cite{}.
% 
Overall, the proposed models and the maximum likelihood estimation results provide credible mechanisms to understand the development of economic systems.

%% Further developments and investigation
A set of hypotheses have been assumed to formulate parsimonious population dynamic models, which could be relaxed in future work to leverage discrimination between the processes investigated. 
% 
In particular, the mean field assumptions that economic activities similarly interact, proceed to spatial transfers and transfer capital between each other with equal rates, are strong.
% 
Biological organisms interact through interaction networks \cite{Bascompte2003}, and  such structured linkages have been also documented in economic networks  \cite{C.A.HidalgoB.Klinger,Bustos2012,Saavedra2009a}. In particular \cite{C.A.HidalgoB.Klinger} suggests that economic activities are related to one another through a network of relatedness. This relatedness network could be integrated in $\modalphan$, $\modalphap$ and $\moddelta$, weighting the interactions and transfers by the relatedness of two activities.
% 
Similarly, spatial interactions are likely to depend on spatial proximity between countries, although this proximity may not only be related to geographical distance but also modulated by cultural proximity \cite{Bahar2013}.
% 
Last but not least, the strength and directionality of e.g. interactions may temporally change throughout the development of economic activities \cite{Pistorius1997}, while we assumed that the parameters are constant through time.
% 
More developments will be required to asses the importance and consequence of these assumptions. The investigation of more complex models may nevertheless require more detailed and richer time series, in order to extract the extra information required to constrain the additional parameters.

%% General summary and conclusion
In conclusion, our study fosters a biologically inspired approach to understand the mechanisms shaping the long-term dynamics of economic systems.
% 
Our results quantitatively support that, akin to the major forces influencing the dynamics of ecosystems, ecological interactions, spatial transfers and economic activity transformations play an important role in the development of economic systems. 
% 
The quantitative paradigm to study economic systems has mainly relied on tools and models borrowed from physics, which assume a world in equilibrium \cite{sornette2014physics}.
% 
Not only bringing insightful concept but also providing quantitative tools, it may well be that evolutionary biology, focusing on dynamical processes and emergence, is now better equipped to understand the fundamental forces shaping the evolution of economic systems.
% 
%, and address the critiques addressed to mainstream economics \cite{XXX}.
%
% 
% Although the foundations of the analogies between inert matter and economic systems were early on criticized \cite{Veblen1898},  Here, we demonstrate how the application of a model inspired from biological processes, can provide insight on economic growth. 



% Its application to economic systems could provide a better understand of economic development, and may suggest biologically inspired economic policies to overcome fundamental issues arising from misappropriate regulations. We could surely get insights from ecological systems, that have survived major environmental crisis for more that 3.5 billion years.

\section{Acknowledgements}
% The authors thank Thomas Poulet and François Duchene for helpful discussions and comments.
L.P. and V.B. were supported by the SNF grant 310030E\_205556.

\section{Code availability}
The simulation code is available on request.

\printbibliography[heading=subbibliography]