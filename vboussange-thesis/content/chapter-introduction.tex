% !TEX root = ../my-thesis.tex
%
\chapter{Introduction}
\label{sec:intro}

% \cleanchapterquote{Le bout du monde et le fond du jardin contiennent la même quantité de merveilles.}{Christian Bobin}{(French poet)}

\subsubsection*{Biological and economic systems as complex adaptive systems}
%% definition of complex adaptive system
What are the similarities between biological and economic systems? Both are complex adaptive systems (CAS): they are composed of heterogeneous entities structured at different levels of organizations, that interact in nonlinear ways and experience evolutionary processes \cite{Levin}. The processes of interaction and evolution operate at one organizational level and result in emergent properties at a higher organisational level. In other words, they result in non-trivial properties that cannot be anticipated from the sole observation of individual entities.
% Example : biology
For example, in biological systems, the disproportionately high biodiversity (i.e., number of species) observed in mountane regions \cite{Rahbek2019} -- at the continental level -- arises from key dynamical processes operating at the population level \cite{Rangel2018}.
%% Example: economics
In economic systems, the biomodal distribution of international income levels -- at the world level -- is partly explained by structures of interactions across economic activities at the regional level \cite{C.A.HidalgoB.Klinger}.
%% limitations of the current understanding
The understanding of the coupling between processes operating at different organizational level is identified as a major frontier in the 21st century science \cite{Strogatz2001a}. A major concern in this direction is to comprehend the necessary set of constraints, with respect to how individual entitities interact and evolve, determining emergent properties. 
% conclusion
The close interplay between ecological and evolutionary processes plays a fundamental role in the emergence of patterns in biological systems \cite{Pelletier2009}, and may influence the dynamics of economic systems \cite{Hodgson2019}.

\begin{figure}
    \centering
    \begin{tikzpicture}[
        node distance=2cm,
        on grid,
        very thick]
    \tikzstyle{node_cust}=[draw, minimum height=0.7cm, minimum width=3.5cm]

    % biological systems
    \node[node_cust, align=center] (genes) {Genes / genomes};
    \node[node_cust, below=of genes] (phen) {Phenotypes};
    \node[node_cust, below=of phen] (pop) {Populations};
    \node[node_cust, below=of pop] (com) {Communities};
    \node[node_cust, below=of com] (ecos) {Ecosystems};
    % biological systems
    \node[node_cust,right=4.5cm of genes] (orga) {Organizational routines};
    \node[node_cust,below=of orga] (Firms) {Firms};
    \node[node_cust,below=of Firms] (ecoa) {Economic activities};
    \node[node_cust,below=of ecoa] (nateco) {National economies};
    \node[node_cust,below=of nateco] (weco) {World economy};
\end{tikzpicture}
\caption{Schematic diagram of proposed organisational levels in biological and economic systems. \textbf{A} is inspired from \cite{Hendry+2016}}
\end{figure}



\subsubsection*{Eco-evolutionary feedbacks}
% 
% History of the understanding of forces in ecology and evolution
The interplay between ecological and evolutionary processes critically affects the dynamics of biological systems \cite{Pelletier2009}.
% 
%% specific focus on ecology and evolution
Since Darwin, it is genuinely acknowledged that natural selection, as the result of biotic and abiotic interactions, determines the survival of phenotypes, and therefore that ecological change affects evolutionary response \cite{Ezard2009}.
% 
For instance, Darwin realised that the finches in the Galapagos island evolved varied beaks because of the variations in ecological opportunities across the islands \cite{darwin2004origin}.
% 
%% Recent realisation of eco-evolutionary dynamics
Nonetheless, a time scale separation between the ecological and evolutionary processes has traditionally been assumed \cite{slobodkin1980growth}, so that the fact that ecological processes, such as population growth, may be affected by evolutionary processes has only be recently investigated. Empirical studies have now demonstrated that evolution can be rapid and occur on similar time scales as ecology \cite{Hairston2005} and have quantifiable effect on ecological dynamics \cite{Ezard2009}, leading to eco-evolutionary feedbacks. 
% 
% Example of eco-evolutionary dynamics
Eco-evolutionary feedbacks involve situations where an ecological property influences evolutionary change, which then feeds back to an ecological property, or vice versa \cite{Govaert2019a}. For instance, \cite{Dieckmann1999} shows that feedbacks between population density and trait evolution are sufficient to explain speciation via evolutionary branching.
% 
% Empirically, \cite{XXX} shows that in the population of XX, XX happened.
% 
%% Conclusion
Eco-evolutionary feedbacks must be accounted for in order to understand the mechanisms driving the dynamics of ecosystems \cite{Govaert2019a}. Because they are affected by analogous processes, this realisation should also apply to economic systems.

\subsubsection*{An urgent need for better ecosystem models}
The effect of direct anthropogenic pressure, together with climate change, is rapidly affecting ecosystems \cite{Ellis2011,Midgley2019}. Ecosystems are approaching state shifts \cite{Barnosky2012}, which in turn will greatly affect human societies \cite{Mooney2009}.
%
%% Constatation of system state shift
Current extinction rates are higher than would be expected from the fossil record \cite{Barnosky2011}. Based on habitat models, \cite{Midgley2019} predicts on the basis of mid-range climate-warming scenarios for 2050 that 15\% to 37\% of species would be committed to extinction. 
% 
%%
While there is a general agreement that anthropogenic pressure and climate change will have a negative effect on the biosphere \cite{fischlin2007ecosystems}, their precise effect on ecosytem dynamics is unclear. For instance, with global warming, species are likey to shift towards higher elevations and higher latitudes \cite{Chen2011}. Because the speed of range shifts differ between different ecological groups, climate change is expected to modify the current organization of trophic interactions \cite{Descombes2020}, affecting ecosystem functioning.
%
%% 
Current forecasts of ecosystem states are based on habitat models, where species habitats are learnt from species occurence data, and are reprojected it given environmental predictors.
% 
Such approaches miss the processes of ecological interactions, evolutionary change and species dispersal \cite{Pearson2003}, that are expected to play a critical role in the evolution of the biosphere in the coming decades \cite{Norberg2012}.
% 
In order to mitigate the consequences of human development, it is of utmost urgency to better understand the mechanisms influencing the biosphere \cite{Rahbek2019}, and utilize this knowledge to provide forecasts to designing adequate management of future ecosystem services \cite{Clark2001,Norberg2012}.

% \subsubsection*{Endogenous forces in economic systems}
% Traditional approaches to economics assume the rationality of economic agents \cite{XXX}. Economic dynamics
% % 
% In contrast, evolutionary economics 

\subsubsection*{Forward modelling of eco-evolutionary processes}
Eco-evolutionary processes are difficult to observe in biological systems, and conducting controlled experiments to quantify their roles is usually not possible \cite{Pontarp2019}. As such, a deductive method relying on forward modelling has traditionally been used to investigate the effect of eco-evolutionary processes \cite{Brummitt2020}. Along this approach, hypotheses are embedded in a model -- so called mechanistic model \cite{XXX} --, wich forward integration generates predictions. The resulting qualitative dynamics and/or quantitative predictions are validated against common intuition and empirical data, and further refined to elaborate a theory \cite{Sayama,Brummitt2020,Schmidt2009}.
% 
%% early mathematical models
In the early 1930s to 1940s, by formulating tractable mathematical models, the work of Fisher, Wright and Haldane has greatly contributed to the modern synthesis \cite{huxley1942evolution}, generally accepted as the pillar of our current understanding of evolutionary dynamics. Nonetheless, the requirement of tractable mathematical models has involved strong assumptions, such as simplified ecological scenarios, that are not representative of the complexity of eco-evolutionary feedbacks in nature \cite{Govaert2019a}.
% 
%% computers
With the increase in computational capacity, novel approaches to forward modelling have appeared, heavily relying on individual based models (IBMs) \cite{XXX}. IBMs allow the forward integration of complex hypothesis while requiring very little simplifying assumptions \cite{XXX}. While offering the possibility to investigate more realistic scenarios, their lack of analytical traceability may occult the mechanisms generating the emergent properties.
% 
%% computers and analytical framework
Adaptive dynamics theory \cite{Metz1995}, together with recent mathematical techniques \cite{Meleard,Nordobtten,Lion}, are providing tools for the analytical underpinning to IBM simulations under simplified scenarios \cite{XXX}. 
% 
% They allow to derive analytical expressions for the population macroscopic properties (e.g., population size and trait variance) from individual-based assumptions. 
% 
Analogous to renormalisation group analysis developed in quantum and statistical physics, they provide -- in combination with numerical simulations -- the appropriate modelling framework to obtain a general understanding of the mechanisms generated by eco-evolutionary dynamics \cite{Levin2002,Govaert2019a}.
% 
This combination of numerical simulations and analytical insights has successfully shed new lights, for instance, on the effect of environmental feedbacks and on the emergence of polymorphism under frequency-dependent disruptive selection \cite{Dieckmann1999,Doebeli2003}.
% 
While we have a good understanding on the effect of eco-evolutionary processes on population dynamics in simple landscapes \cref{Mirrahimi}, it is however unclear whether such results hold under complex landscapes, such as those observed in mountains \cref{Rahbek2019}.
% 
Furthermore, the predictive ability of mechanistic modesl has remained poor \cite{DeAngelis2015}, due to a low pervasion of observation data in mechanistic models \cite{XXX}.
% This understanding uses matrices or networks to create representations of complex systems that do not ignore the identity of the elements involved or their interactions. These ideas, which are now prevalent in fields such as machine learning and physics, have begun to make their way into economics under the umbrella of economic complexity

\subsubsection*{Machine learning, inverse modelling and scientific machine learning}


\subsubsection*{Programming languages}


\subsubsection*{Thesis outline}



\newpage

remains to be understood, and is more important than ever in a rapidly changing world.

\section{Complex adaptive systems}
\label{sec:intro:cas}
\begin{outline}
    \1 A central aim in the discipline of Ecology is to determin the underlying causes of variation in the abundance and sitribution of species.
    \1 Ecological and economic systems are complex adaptive systems (CAS): they are systems that are composed of many entities with heterogeneous characteristics, which interact and experience selection processes. Those processes act at the individual level, but are key in determining the macroscopic behaviour at the system level, a feature that make those systems unique.
    \1 Complex interconnected systems pose a major challeng to scientific study in ecology and economics \cite{Ye2016} (and references therin).
        \2 the common approach of reducing these systems to linearly independent components overlooks important interactions for the sake of computational tractability 
        \2 statistical frameworks (e.g., PCA, GLM, multivariate autoregressive models), assume that caysal factors do not interact with each other and have independent or additive effects on a response variable,
            \3 simplification leads to probelms in identifying associations (refs 5-6 of \cite{Ye2016})
            \3 cannot predict out-of sample behaviour
        \2 complex equation-based modeol explicitly accounting for each interaction have great intuitive appeal 
            \3 but those models suffer from their many parameters to be precisely determined given the available data (curse of dimensionality (ref 9 \cite{Ye2016}))
            \3 problem is amplified because in biological fields the relevant units may not behave according to the fundamental equations.
\end{outline}
\subsubsection{Biological systems}
\begin{outline}
    \1 Biodiversity results from a hierarchy of processes acting at different scales of time and space. Variations experienced by organisms, their interactions between them and with the environment, and selection pressure acting upon groups of organisms are of particular relevance for explaining differences in species richness at the ecosystem levels.
            \2 The synthetic  theory of evolution (see e.g. Gayon 2003): with genetics (Mendel) and DNA (James Watson and Francis Crick)
            \2 \textit{Nothing in biology makes sense except in the light of evolution} (Dobhansky 1973)
    \1 explanation for the main principles underlying the emergence of biodiversity: mutliple processes that interact at different scale in space and time 
            \2 allopatric speciation
            \2 ecological speciation 
            \2 dispersal
            \2 adaptation
            \2 those processes interact simulatneously withing the surrounding environment
    \1 Traits: measurable characteristics that reflect and shape evolutionary history (Darwin 1859). Natural selection promotes the evolution of traits thatoptimize species survival under specific environmental conditions..
\end{outline}
\subsubsection{Economic systems}
\begin{outline}
    \1 The economic trajectory of a country is greatly affected by the ensemble of economic actors and their interactions, that structure its economy. Firms are adaptive entities that respond to the environment in which they operate according to their characteristics, that vary over time. By interacting together and experiencing selection pressure, they determine economic growth at the country level.
        \2 \textbf{Universal Darwinism}
\end{outline}
\subsubsection{Research questions}
\begin{outline}
    \1 Despite the intrinsic variability of the entities that compose them, and despite the complexity of the processes driving their dynamics, regularities at the macroscopic level emerge in ecological and economic systems. This is the case of large-scale spatial patterns of biodiversity and differences in economic growth across countries, calling for a mechanistic understanding of the essential mechanisms that generate them.
        \2 Multiple arrangements of parts that result in a complex set of effects in a system are defined as mechanisms (Dawkins 2010)
\end{outline}

\section{Eco evolutionary processes}
\begin{outline}
    \1 Eco-evolutionary processes and analogous economic processes acting upon firms have been proposed to play a major role in the emergence of macroscopic patterns in ecological and economic systems. Nonetheless, a quantitative investigation of their importance is missing.
        \2 The interplay between ecological processes, the processes that regulate interactions between organisms, and evolutionary processes, the change of the characteristics of biological populations over time, has recently received increasing attention to explain current biodiversity patterns. 
        \2 Analogous economic processes have been proposed to explain differences in economic growth across nations.
    \1 A quantitative investigation of how those patterns can emerge from eco-evolutionary processes is required to improve our current understanding and generate a parsimonious theory with predictive power. This defines the goal of this project, which undertakes this investigation through a unique approach that consists in confronting quantitative eco-evolutionary models to empirical data.
\end{outline}

\section{Models and challenges}
\begin{outline}
    \1 Eco-evolutionary models are complicated and necessitate the use of computers to be simulated and analysed against data. This poses a number of methodological challenges that we adress in the first part of this project.
        \2 Entities in CAS have distinct quantitative attributes that determine their fitness in a given environment. Accounting for the variety of these characteristics leads to models with high dimensionality, associated to a high if not prohibitive computational cost preventing its simulation.
            \3 The model zoo
                \4 Agent Based model: hard to scale up
                \4 PDE: hard to scale up
                \4 In particular, partial differential equation (PDE) models, which can encode eco-evolutionary processes acting upon entities defined by many characteristics, are cursed by their dimensionality.
                \4 Machine learning: scale up
            \3 To this aim, we develop machine learning algorithms that break down the curse of dimensionality by relying on neural networks to approximate the solution to PDE models.
        \2 An other difficulty consists in confronting eco-evolutionary models with data, since those models cannot be manipulated by standard statistical techniques. 
            \3 We apply methods commonly employed in the training of neural networks, together with model selection techniques, to infer from candidate models fundamental mechanisms that characterise the patterns under investigation.
    \1 The machine learning approximations that we develop allow for efficient model simulations, that we combine with training techniques and model selection methods to explore the motivated research question.
\end{outline}

\missingfigure{Here you could add a coneptual figure, similar to Florian Patout (see evernote), that shows the interplay between selection and variation.}

\section{Machine learning : opportuntities}
\begin{outline}
    \1 State of the art machine learning techniques have yielded transformative results across divers scientific disciplines [REF], but rely on a large amount of data [REF], while environmental sciences rely in a small data regime where those techniques are typically not suited \cite{Raissi2019a}. Recently, physics informed machine learning has emerged as a tool to constrain fully parametric methods with scientific knowledge, for data efficiency and extrapolation \cite{Raissi2019a}. The key idea is to refine the learning with scientific knowledge by adding additional constraints in the objective function, given by ODEs/ PDEs models.
    \1 \cite{Karpatne2017}
    \1 \cite{Rolnick2023}, Tackling Climate Change with Machine Learning: Changes in climate are increasingly affecting the distribution and composition ofecosystems. This has profound implications for global biodiversity, as well as agriculture, disease, and natural re- sources such as wood and fish. ML can help by supporting efforts to monitor ecosystems and biodiversity.
    Monitoring
\end{outline}

\section{Learning from models}
\begin{outline}
    \1 we develop quantitative models that embed general eco-evolutionary processes, and test them against data to explore hypotheses on the fundamental mechanisms that drive patterns of biodiversity and economic growth.
        \2 From one hand, we explore how eco-evolutionary processes, in combination with complex landscape topologies, can explain patterns of species diversity.
        \2 To this aim, we develop and analyse an eco-evolutionary model on spatial graphs, to understand how the combination of eco-evolutionary processes and complex landscapes might have shaped biodiversity patterns that are found in complex landscapes such as mountain regions.
        \2 On the other hand, we investigate how eco-evolutionary processes can provide new insights in the understanding of economic dynamics.
        \2 We proceed by developing a simple eco-evolutionary model which explanatory power we test against long time series that capture the dynamics of asset size of economic sectors.
    \1 Overall, this project is a step towards providing a useful conceptualisation of fundamental eco-evolutionary mechanisms that shape the features of the world that surrounds us.
\end{outline}


\section{Thesis outline}
\label{sec:intro:structure}

\textbf{Part \ref{part:I}\\
An eco-evolutionary model on spatial graphs} \\[0.2em]
It is not clear how landscape connectivity and habitat heterogeneity influence differentiation in biological populations. 
%
To obtain a mechanistic understanding of underlying processes, we construct an individual-based model that accounts for eco-evolutionary and spatial dynamics over graphs. 
%
Individuals possess both neutral and adaptive traits, whose co-evolution results in differentiation at the population level.
%
In agreement with empirical studies, we show that characteristic length, heterogeneity in degree and habitat assortativity drive differentiation.
%
By using analytical tools that permit a macroscopic description of the dynamics, we further link differentiation patterns to the mechanisms that generate them.
%
This part provides support for a mechanistic understanding of how landscape features affect diversification.

\textbf{Part \ref{part:II}\\
Scientific machine learning for eco-evolutionary modelling} \\[0.2em]
% Mechanistic models crystallise hypothesis into a synthetic framework that allows for a description of mechanisms driving the dynamics of complex adaptive systems.
%
It is a daunting task to obtain an agreement between mechanistic models and real world systems. In particular, there is a need to account for the dimensionality of the evolutionary and spatial structures over which agents interact and evolve. Furthermore, the calibration of such models is difficult.
% given that direct measurements to estimate quantities of interest are in general not possible, and only a small set of undirect observations are available.
%
To adress the difficulties that arise due to the dimensionality of models, we develop two numerical methods to solve high-dimensional non-local nonlinear PDES that arise in eco-evolutionary models. We implement those methods in a software, \texttt{HighDimPDE.jl}, that integrates within an open source ecosystem for Scientific Machine Learning in the Julia programming language.
%
We further present a scheme to estimate the parameters of a mechanistic model from empirical data sets. We show with analytical arguments that the use of different shallow time series allows for a better estimation than a unique, possibly deeper time series.
%
This part provides ready-to-use modeling tools to adress the intrinsic complexity of complex adaptive systems.

\textbf{Part \ref{part:III}\\
Briding eco-evolutionary models and data} \\[0.2em]
Despite evidences that alike biological systems, economic systems are complex adaptive systems that continuously adapt and experience evolutionary processes, economists have discarded biological models and have rather relied on mechanistic models inspired from physics.
%
Building upon an analogy between economic sectors and biological functional groups, we use a biological model to quantitatively investigate whether eco-evolutionary processes characterise the dynamics of economic sectors.
%
Overall, we find that interactions across economic sectors, evolution of new economic sectors, and international transfers play a major role in the dynamics of economic sectors at the national level. 
%
The significance and the strength of such processes strongly vary across countries and correlate with standard macroeconomic indices such as the Economic Complexity Index.
%
We relate such patterns to documented patterns in ecology and evolution.
%
This part provides a new perspective on the understanding of the dynamics of economic systems.
% 
% The mechanistic framework is inspired from theoreteical biology that is general enough to encompass forces and processes in both economics systems and ecological systems


\begin{comment}
    The text of this thesis/dissertation contains reprints of the material as it appears in journals.
\end{comment}

\printbibliography[heading=subbibliography]