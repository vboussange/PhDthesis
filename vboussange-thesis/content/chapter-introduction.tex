% !TEX root = ../my-thesis.tex
%
\chapter{Introduction}
\label{sec:intro}

% \cleanchapterquote{Le bout du monde et le fond du jardin contiennent la même quantité de merveilles.}{Christian Bobin}{(French poet)}

\section{Context of modelling}
\label{sec:intro:context}
\begin{outline}
    \1 human curiosity: build models of what surrounds us
        \2 Nature has been fascinating since the beginning of humankind.
            \3  Human's curiosity lead us to propose models capturing our belief on how things work. Those were mainly conceptual models.
        \2 The scientific method started with the Enlightment, during the 18th century (Holmes, 1997).
            \3 Isaac Newton became the revered founder of modern Mechanics due to his intuition, gathered by empirical evidence, about a possible mathematical formalization for the law of universal gravitation. \cite{Equations2021}
            \3 20th century: logical empiricism dominated the philosophy of science and sceitnsts searched for fundamental theoretical principles explaing the laws of nature, with physics at the central stage (Okashaa 2002)
            \3 in biology, natural history mainly (classification of living organisms without further questioning).
                \4 why is it that we better understand the motion of planets, or the surface of the moon, than e.g. the mechanisms that drive our fingers? (REF)
                \4 biological world poses obstacles in finding laws: nonlinear processes and complexity of processes and spatial and time scales
        \2 The mathematicalisation of soft science was driven by inspiration from physics
            \3 From the 1700s on, nature has been increas-ingly described by mathematical equations,with differential or difference equationsforming the basic framework for describingdynamics. The use of mathematical equationsfor ecological systems came much later, pio-neered by Lotka and Volterra, who showedthat population cycles might be described interms of simple coupled nonlinear differentialequations. It took decades for Lotka–Volterra-type models to become established, but thedevelopment of appropriate differential equa-tions is now routine in modeling ecologicaldynamics. There is no question that the in-jection of mathematical equations, by forcing“clarity and precision into conjecture”(2), hasled to increased understanding of populationand community dynamics. As in sciencein general, in ecology equations are a keymethod of communication and of framinghypotheses. These equations serve as compactrepresentations of an enormous amount ofempirical data and can be analyzed by thepowerful methods of mathematics \cite{DeAngelis2015}.
        \2 AI era
        \2 Scientific revolution of Darwin thinking, by Kuhn's definition (Dawkins 2010)
        \2 Universal Darwninism
    \1 more than a curiosity : a necessity
        \2 Approaching a state shift in Earth biosphere: \cite{Barnosky2012}
    \1 Bridge those models with data
\end{outline}


\section{Complex adaptive systems}
\label{sec:intro:cas}
\begin{outline}
    \1 A central aim in the discipline of Ecology is to determin the underlying causes of variation in the abundance and sitribution of species.
    \1 Ecological and economic systems are complex adaptive systems (CAS): they are systems that are composed of many entities with heterogeneous characteristics, which interact and experience selection processes. Those processes act at the individual level, but are key in determining the macroscopic behaviour at the system level, a feature that make those systems unique.
    \1 Complex interconnected systems pose a major challeng to scientific study in ecology and economics \cite{Ye2016} (and references therin).
        \2 the common approach of reducing these systems to linearly independent components overlooks important interactions for the sake of computational tractability 
        \2 statistical frameworks (e.g., PCA, GLM, multivariate autoregressive models), assume that caysal factors do not interact with each other and have independent or additive effects on a response variable,
            \3 simplification leads to probelms in identifying associations (refs 5-6 of \cite{Ye2016})
            \3 cannot predict out-of sample behaviour
        \2 complex equation-based modeol explicitly accounting for each interaction have great intuitive appeal 
            \3 but those models suffer from their many parameters to be precisely determined given the available data (curse of dimensionality (ref 9 \cite{Ye2016}))
            \3 problem is amplified because in biological fields the relevant units may not behave according to the fundamental equations.
\end{outline}
\subsubsection{Biological systems}
\begin{outline}
    \1 Biodiversity results from a hierarchy of processes acting at different scales of time and space. Variations experienced by organisms, their interactions between them and with the environment, and selection pressure acting upon groups of organisms are of particular relevance for explaining differences in species richness at the ecosystem levels.
            \2 The synthetic  theory of evolution (see e.g. Gayon 2003): with genetics (Mendel) and DNA (James Watson and Francis Crick)
            \2 \textit{Nothing in biology makes sense except in the light of evolution} (Dobhansky 1973)
    \1 explanation for the main principles underlying the emergence of biodiversity: mutliple processes that interact at different scale in space and time 
            \2 allopatric speciation
            \2 ecological speciation 
            \2 dispersal
            \2 adaptation
            \2 those processes interact simulatneously withing the surrounding environment
    \1 Traits: measurable characteristics that reflect and shape evolutionary history (Darwin 1859). Natural selection promotes the evolution of traits thatoptimize species survival under specific environmental conditions..
\end{outline}
\subsubsection{Economic systems}
\begin{outline}
    \1 The economic trajectory of a country is greatly affected by the ensemble of economic actors and their interactions, that structure its economy. Firms are adaptive entities that respond to the environment in which they operate according to their characteristics, that vary over time. By interacting together and experiencing selection pressure, they determine economic growth at the country level.
        \2 \textbf{Universal Darwinism}
\end{outline}
\subsubsection{Research questions}
\begin{outline}
    \1 Despite the intrinsic variability of the entities that compose them, and despite the complexity of the processes driving their dynamics, regularities at the macroscopic level emerge in ecological and economic systems. This is the case of large-scale spatial patterns of biodiversity and differences in economic growth across countries, calling for a mechanistic understanding of the essential mechanisms that generate them.
        \2 Multiple arrangements of parts that result in a complex set of effects in a system are defined as mechanisms (Dawkins 2010)
\end{outline}

\section{Eco evolutionary processes}
\begin{outline}
    \1 Eco-evolutionary processes and analogous economic processes acting upon firms have been proposed to play a major role in the emergence of macroscopic patterns in ecological and economic systems. Nonetheless, a quantitative investigation of their importance is missing.
        \2 The interplay between ecological processes, the processes that regulate interactions between organisms, and evolutionary processes, the change of the characteristics of biological populations over time, has recently received increasing attention to explain current biodiversity patterns. 
        \2 Analogous economic processes have been proposed to explain differences in economic growth across nations.
    \1 A quantitative investigation of how those patterns can emerge from eco-evolutionary processes is required to improve our current understanding and generate a parsimonious theory with predictive power. This defines the goal of this project, which undertakes this investigation through a unique approach that consists in confronting quantitative eco-evolutionary models to empirical data.
\end{outline}

\section{Models and challenges}
\begin{outline}
    \1 Eco-evolutionary models are complicated and necessitate the use of computers to be simulated and analysed against data. This poses a number of methodological challenges that we adress in the first part of this project.
        \2 Entities in CAS have distinct quantitative attributes that determine their fitness in a given environment. Accounting for the variety of these characteristics leads to models with high dimensionality, associated to a high if not prohibitive computational cost preventing its simulation.
            \3 The model zoo
                \4 Agent Based model: hard to scale up
                \4 PDE: hard to scale up
                \4 In particular, partial differential equation (PDE) models, which can encode eco-evolutionary processes acting upon entities defined by many characteristics, are cursed by their dimensionality.
                \4 Machine learning: scale up
            \3 To this aim, we develop machine learning algorithms that break down the curse of dimensionality by relying on neural networks to approximate the solution to PDE models.
        \2 An other difficulty consists in confronting eco-evolutionary models with data, since those models cannot be manipulated by standard statistical techniques. 
            \3 We apply methods commonly employed in the training of neural networks, together with model selection techniques, to infer from candidate models fundamental mechanisms that characterise the patterns under investigation.
    \1 The machine learning approximations that we develop allow for efficient model simulations, that we combine with training techniques and model selection methods to explore the motivated research question.
\end{outline}

\missingfigure{Here you could add a coneptual figure, similar to Florian Patout (see evernote), that shows the interplay between selection and variation.}

\section{Machine learning : opportuntities}
\begin{outline}
    \1 State of the art machine learning techniques have yielded transformative results across divers scientific disciplines [REF], but rely on a large amount of data [REF], while environmental sciences rely in a small data regime where those techniques are typically not suited \cite{Raissi2019a}. Recently, physics informed machine learning has emerged as a tool to constrain fully parametric methods with scientific knowledge, for data efficiency and extrapolation \cite{Raissi2019a}. The key idea is to refine the learning with scientific knowledge by adding additional constraints in the objective function, given by ODEs/ PDEs models.
    \1 \cite{Karpatne2017}
    \1 \cite{Rolnick2023}, Tackling Climate Change with Machine Learning: Changes in climate are increasingly affecting the distribution and composition ofecosystems. This has profound implications for global biodiversity, as well as agriculture, disease, and natural re- sources such as wood and fish. ML can help by supporting efforts to monitor ecosystems and biodiversity.
    Monitoring
\end{outline}

\section{Learning from models}
\begin{outline}
    \1 we develop quantitative models that embed general eco-evolutionary processes, and test them against data to explore hypotheses on the fundamental mechanisms that drive patterns of biodiversity and economic growth.
        \2 From one hand, we explore how eco-evolutionary processes, in combination with complex landscape topologies, can explain patterns of species diversity.
        \2 To this aim, we develop and analyse an eco-evolutionary model on spatial graphs, to understand how the combination of eco-evolutionary processes and complex landscapes might have shaped biodiversity patterns that are found in complex landscapes such as mountain regions.
        \2 On the other hand, we investigate how eco-evolutionary processes can provide new insights in the understanding of economic dynamics.
        \2 We proceed by developing a simple eco-evolutionary model which explanatory power we test against long time series that capture the dynamics of asset size of economic sectors.
    \1 Overall, this project is a step towards providing a useful conceptualisation of fundamental eco-evolutionary mechanisms that shape the features of the world that surrounds us.
\end{outline}


\section{Thesis outline}
\label{sec:intro:structure}

\textbf{Part \ref{part:I}\\
An eco-evolutionary model on spatial graphs} \\[0.2em]
It is not clear how landscape connectivity and habitat heterogeneity influence differentiation in biological populations. 
%
To obtain a mechanistic understanding of underlying processes, we construct an individual-based model that accounts for eco-evolutionary and spatial dynamics over graphs. 
%
Individuals possess both neutral and adaptive traits, whose co-evolution results in differentiation at the population level.
%
In agreement with empirical studies, we show that characteristic length, heterogeneity in degree and habitat assortativity drive differentiation.
%
By using analytical tools that permit a macroscopic description of the dynamics, we further link differentiation patterns to the mechanisms that generate them.
%
This part provides support for a mechanistic understanding of how landscape features affect diversification.

\textbf{Part \ref{part:II}\\
Scientific machine learning for eco-evolutionary modelling} \\[0.2em]
% Mechanistic models crystallise hypothesis into a synthetic framework that allows for a description of mechanisms driving the dynamics of complex adaptive systems.
%
It is a daunting task to obtain an agreement between mechanistic models and real world systems. In particular, there is a need to account for the dimensionality of the evolutionary and spatial structures over which agents interact and evolve. Furthermore, the calibration of such models is difficult.
% given that direct measurements to estimate quantities of interest are in general not possible, and only a small set of undirect observations are available.
%
To adress the difficulties that arise due to the dimensionality of models, we develop two numerical methods to solve high-dimensional non-local nonlinear PDES that arise in eco-evolutionary models. We implement those methods in a software, \texttt{HighDimPDE.jl}, that integrates within an open source ecosystem for Scientific Machine Learning in the Julia programming language.
%
We further present a scheme to estimate the parameters of a mechanistic model from empirical data sets. We show with analytical arguments that the use of different shallow time series allows for a better estimation than a unique, possibly deeper time series.
%
This part provides ready-to-use modeling tools to adress the intrinsic complexity of complex adaptive systems.

\textbf{Part \ref{part:III}\\
Briding eco-evolutionary models and data} \\[0.2em]
Despite evidences that alike biological systems, economic systems are complex adaptive systems that continuously adapt and experience evolutionary processes, economists have discarded biological models and have rather relied on mechanistic models inspired from physics.
%
Building upon an analogy between economic sectors and biological functional groups, we use a biological model to quantitatively investigate whether eco-evolutionary processes characterise the dynamics of economic sectors.
%
Overall, we find that interactions across economic sectors, evolution of new economic sectors, and international transfers play a major role in the dynamics of economic sectors at the national level. 
%
The significance and the strength of such processes strongly vary across countries and correlate with standard macroeconomic indices such as the Economic Complexity Index.
%
We relate such patterns to documented patterns in ecology and evolution.
%
This part provides a new perspective on the understanding of the dynamics of economic systems.
% 
% The mechanistic framework is inspired from theoreteical biology that is general enough to encompass forces and processes in both economics systems and ecological systems


\begin{comment}
    The text of this thesis/dissertation contains reprints of the material as it appears in journals.
\end{comment}