% !TEX root = ../my-thesis.tex
%
\chapter{CV}
\label{sec:CV}

\section*{Personal Information}

\section*{Personal skills}

\section*{Education}
\begin{experiences}
    \experience
      {10.2022}   {Ph.D in Environmental Sciences}{Swiss Federal Institute for Forest, Snow and Landscape (WSL | Swiss Federal Institute of Technology Zurich, ETH)}{Switzerland}
      {09.2018} {"Forward and inverse modelling of eco-evolutionary processes".
                      %   \begin{itemize}
                      %     \item Part I: "\textit{Neutral and adaptive diversification in spatial graphs}"                      
                      %     \item Part II: "\textit{Scientific Machine Learning with applications to eco-evolutionary modeling}"                
                      %     \item Part III: "\textit{Econobiology: understanding economic dynamics with biological models}"   
                      %   \end{itemize}
                      Under the guidance of Prof. Dr. Loïc Pellissier.
                      }
                      {computational biology, mathematical modeling, scientific machine learning, complex systems, complexity economics}
    \emptySeparator
    \experience
      {06.2017} {Full year academic exchange}{University of New South Wales (UNSW Sydney)}{Australia}
      {09.2016} {}{computational methods for finance, electrical energy, chemical reaction engineering}
      % computational methods for finance, electrical energy, chemical reaction engineering
    \emptySeparator
    \experience
      {06.2017}{Master thesis in theoretical geomechanics}{UNSW Sydney | CSIRO}{Australia}
      {02.2017}    {"Numerical continuation and bifurcation analysis for unconventional geomechanics". Under the guidance of Dr. Thomas Poulet.
                      }
                      {theoretical geomechanics, numerical continuation, bifurcation analysis}
    \emptySeparator
    \experience
    {08.2018}       {B.S./ M.S. in Energy and Environmental Engineering}{Institut National Des Sciences Appliquées de Lyon (INSA Lyon)}{France}
    {09.2013} {
    \begin{itemize}
        \item Two-year undergraduate intensive course in mathematics and physics.
  Ranking : 21/650 students.
        \item Three-year undergraduate engineering course in Energy and Environmental Systems, focused on Advanced Energy Systems and Efficiency.
    \end{itemize}
    }{fluid mechanics, thermodynamics, electrical networks and optimisation, energy markets}
  \end{experiences}

\section*{Professional appointments}
\begin{experiences}
    \experience
      {08.2018}   {R\&D intern}{Compagnie National du Rh\^one (CNR)}{France}
      {03.2018} {Development of an Energy Management System based on various optimisation techniques for optimal production of renewable resources. Applications to EU sponsored projects:
      \begin{itemize}
          \item \hyperlink{https://www.jupiter1000.eu/english}{Jupiter1000 (power-to-gas)}
          \item \hyperlink{https://www.cnr.tm.fr/en/innovation/close-to-the-pulse-of-the-territories/}{Move in pure (vehicle-to-grid)}
          \item \hyperlink{https://www.youtube.com/watch?v=962bBweyx1s}{Marie-Galante island (micro-grid)}
      \end{itemize}}
                      {software development, mathematical optimisation, energy trading}
  \end{experiences}

\section*{Publications}

\section*{Talks}

\section*{Softwares}

\begin{projects}
	\project
	{MiniBatchInference.jl}{2022}
	{\github{vboussange/MiniBatchInference.jl}}
	{A Julia package for maximum likelihood estimation and model selection of strongly nonlinear dynamical models.}
	{Julia}
	
	\project
	{HighDimPDE.jl}{2021}
	{\github{vboussange/HighDimPDE.jl} }
	{A Julia package that breaks down the curse of dimensionality in solving non local, non linear PDEs.}
	{Julia}
				
	\project
	{EvoId.jl}{2019 - 2021}
	{\github{vboussange/EvoId.jl} }
	{Evolutionary individual based modelling, mathematically grounded.}
	{Julia}
	
    \project
	{OptiVPP}{2018}
	{\website{}{confidential}}
	{Energy Management System for Virtual Power Plants.}
	{Python, GAMS}
	
	\textbf{Open Source contributions}\\
	SciML, DiffEqFlux.jl, CUDA.jl, Flux.jl, LightGraphs.jl.
\end{projects}

\section*{Teaching and supervision}

\section*{Reviews}
