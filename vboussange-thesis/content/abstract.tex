% !TEX root = ../my-thesis.tex
%
\pdfbookmark[0]{Summary}{Summary}
\addchap*{Summary}
\label{sec:summary}

\begin{outline}
    % General introduction
    \1 Since life emerged 4 billion years ago, its complexity has evolved.
        \2 Complex Systems (CS) are generally defined as a category of dynamical systems composed of many individual entities, be they biological, socio-cultural or economic, spatially organised and interacting locally in a nonlinear way. The adjective Adaptive is used to define CS which are subjected to evolutionary mechanisms (Levin [2002]). These include the biosphere, socio--cultural systems and economical systems. The agents adapt to local conditions and are subjected to selection processes acting at a macro level.
    
    % Main content
    \1 In this thesis, a novel framework for bridging mechanistic models of CAS and data is presented.
        \2 An eco-evolutionary model of interacting organisms is theoretically investigated 
            \3 to understand the emergence of biodiversity in complex landscapes

        \2 A set of tools are developped to 
        
        \2 Those tools are used to investigate the processes that drive the macroscopic dynanmics of economies across countries

    % Discussion
    \1 It is shown that bridging theoretical models and data deepens our current understanding of processes and can brings a new perspective.
        \2 Bridging disciplines provides a remarkably clear understanding of universal mechanisms that have shaped the economics dynamics.

    % Outline
    \1 This thesis moves beyond the dichotomy between theoretical and data science approaches and provides a novel framework for formalizing and exploring multiple hypotheses and reconstructions associated with the processes that drive CAS. Model comparision with empirical data serves as hindcast, which might inform evolutionary trajectories. By advancing our understanding on the processes that dictate the dynamics of CAS, we can better anticipate the radical changes that we will face in the next decades
\end{outline}

% \vspace*{20mm}

{\usekomafont{chapter} Résumé}
\label{sec:summary-fr}

\begin{outline}
    % General introduction
    \1 Same as above, but in french
\end{outline}