% !TEX root = ../my-thesis.tex
%
\pdfbookmark[0]{Summary}{Summary}
\addchap*{Summary}
\label{sec:summary}
\small{
Biological and economic systems are complex adaptive systems, composed of multiple organisms and entities in interaction, themselves experiencing evolutionary processes.
% 
Interaction and evolutionary processes operate at different organizational scales, from genes to ecosystems and from individual behavior rules to national economies, generating complex couplings across scales. Yet, despite this astonishing organizational complexity, biological and economic systems both display invariant patterns. 
% 
% This is the case of spatial patterns of biodiversity and economic growth.
%
The latter originate from general organizational principles, and the study of ecological and economic systems deals with identifying them.

Recently, studies have shown that evolutionary processes can occur on similar time scales as ecological processes, generating eco-evolutionary feedbacks which may play an important role on the dynamics of biological systems. % In particular, the interplay between population dynamics, dispersal and mutations affects the phenotypic evolution of populations, and can facilitate or prevent populations to adapt to local environment conditions. 
% 
Also, in economic systems, studies are suggesting that economic change is determined by analogous eco-evolutionary processes. 
%  evolutionary economics suggests that economic activities behave as autonomous entities.
% Because of rapid environmental disruptions affecting ecosystems, the eco-evolutionary feedbacks involved in adaptation mechanisms are expected to affect ecosystem dynamics in the coming decades, but
Yet, our understanding of eco-evolutionary processes and feedback mechanisms in empirical systems is limited, because of the over simplicity of current eco-evolutionary models and the lack of confrontation with empirical data. %Current eco-evolutionary models do not capture important features of empirical systems, such as the complexity of their spatial structures, and the variety of their characteristics. Empirical data, carrying signatures of organizational principles, could be used to advance our understanding, but the information it contains is difficult to extract.
% 
% Aiming at advancing our understanding of eco-evolutionary dynamics in empirical systems
% 
% Aiming at identifying organizational principles in biological and economic systems
% 
Developing novel modelling approaches to improve the modelling of empirical systems, this thesis aims at advancing our general understanding of eco-evolutionary processes and feedbacks shaping the dynamics of biological and economic systems.

Specifically, \cref{\chapi} presents and analyses an eco-evolutionary model on spatial graphs to understand how eco-evolutionary processes, together with complex habitat structures, influence the phenotypic distribution of biological populations. \Cref{\chapii} presents an inverse modelling framework to estimate the most likely parameter values of dynamical models from empirical data, permitting to discriminate between competing eco-evolutionary hypotheses. \Cref{\chapiv} test whether processes operating on economic activities, comprising positive and negative interactions between them, their spatial dispersal and their transformations, can explain their dynamics at the country level. To reach this goal, the inverse modelling framework, together with data covering 59 years of economic time series over 74 countries, are used. Finally, \Cref{\chapiv} presents two numerical methods to efficiently simulate eco-evolutionary models of biological populations structured in high dimensional spatial and phenotypic spaces.
% , allowing to capture mechanisms associated to the variety of properties characterizing real-world systems

Together, this thesis advances our general understanding on the eco-evolutionary processes and feedbacks shaping the dynamics of biological populations and economic activities. A holistic map of elemental eco-evolutionary feedbacks influencing spatially structured biological populations is established. As regards economic systems, processes involving positive interactions between economic activities, and their spatial dispersal, are evidenced to systematically affect their dynamics at the country scale.
% 
In parallel to those fundamental results, novel forward and inverse modelling methods are developed, allowing to better capture the dynamics of empirical systems.
% 
In the face of the ongoing climate and biodiversity crisis, it is of urgent to accelerate our general understanding of the mechanisms shaping our world.
% 
Bridging biology, mathematical modelling, machine learning and economics can massively help us to reach this goal.
}

\vspace*{20mm}

{\usekomafont{chapter}Résumé}
\label{sec:summary-fr}
\vspace*{15mm}

\small{
\noindent Les systèmes écologiques et économiques sont des systèmes complexes adaptatifs, composés d'organismes hétérogènes et d'entités en interaction, eux-mêmes affectés par des processus évolutifs. Les processus d'interactions et d'évolution opèrent à différents niveaux d'organisation, des gènes aux écosystèmes et des règles de comportement individuel aux économies nationales, générant des couplages d'échelles. Pourtant, malgré cette incroyable complexité, les systèmes écologiques et économiques démontrent tous deux des comportements invariants. Des mécanismes d'organisation généraux sont à l'origine de ces derniers, et l'enjeu principal de l'étude des systèmes écologiques et économiques est de les identifier.
 
Récemment, des études ont montré que certains processus évolutifs peuvent agir à des échelles de temps similaires à celles des processus écologiques, donnant alors lieu à des boucles de rétroaction éco-évolutives qui pourraient jouer un rôle fondamental sur la dynamique des systèmes écologiques. Par ailleurs, dans les systèmes économiques, des études suggèrent que le changement économique est déterminé par des processus éco-évolutifs analogues. Néanmoins, notre compréhension des processus éco-évolutifs et des boucles de rétroaction dans les systèmes empiriques est limitée, du fait de la simplicité des modèles actuels, et à cause du manque de confrontation avec des données empiriques.  Développant de nouvelles approches visant à mieux modéliser les systèmes empiriques, cette thèse a pour objectif d'avancer dans notre compréhension générale des processus et boucles de rétroaction façonnant la dynamique des systèmes écologiques et économiques.
 
Le \cref{\chapi} présente et analyse un modèle éco-évolutif sur graphes spatiaux pour comprendre comment des processus éco-évolutifs, en concert avec des structures d'habitat complexes, influencent la distribution phénotypique de populations biologiques. Le \cref{\chapii} présente une méthode de modélisation inverse pour estimer la valeur la plus probable de paramètres de modèle dynamiques à partir de données empiriques, permettant de distinguer différentes hypothèses éco-évolutives. Le \cref{\chapiii} teste si des processus s'appliquant aux activités économiques, comprenant des interactions positives et négatives entre elles, leurs dispersions spatiales et leurs transformations, peuvent expliquer leur dynamique à l'échelle d'un pays. Dans ce but, la méthode de modélisation inverse, ainsi que des données économiques couvrant 59 années dans 74 pays, sont utilisées. Finalement, le \cref{\chapiv} présente 2 méthodes numériques permettant de simuler des modèles éco-évolutifs de populations biologiques structurées dans des espaces géographiques et phénotypiques de grande dimension.
 
Cette thèse fait progresser notre compréhension générale des processus et boucles de rétroaction éco-évolutifs impliqués dans la dynamique de populations biologiques et dans celle des systèmes économiques. Une carte des boucles de rétroaction éco-évolutives affectant les populations biologiques structurées est établie. De même, il est démontré que des interactions positives entre les activités économiques, et leur dispersion spatiale, affectent considérablement et de façon généralisée leur dynamique à l'échelle d'un pays. En parallèle de ces résultats fondamentaux, de nouvelles méthodes de simulation et de modélisation inverse sont présentées, permettant de mieux rendre compte des dynamiques de systèmes empiriques. Face à la crise climatique et de la biodiversité en cours, il est urgent d'accélérer notre compréhension générale des mécanismes qui affectent notre monde. L'association de disciplines telles que la biologie évolutive, la modélisation mathématiques, l'apprentissage machine et les sciences économiques peut nous y aider de façon substantielle.}

