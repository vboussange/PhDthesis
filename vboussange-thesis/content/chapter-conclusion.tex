% !TEX root = ../my-thesis.tex
%
\newcommand{\chapi}{\cref{diff-in-graphs}}
\newcommand{\chapii}{\cref{diff-in-graphs}}
\newcommand{\chapiii}{\cref{diff-in-graphs}}
\newcommand{\chapiv}{\cref{diff-in-graphs}}

\chapter{Discussion}
\label{sec:conclusion}

\cleanchapterquote{Les données pertinentes détiennent les réponses.}{French anagram}{}


\section{Contributions}
The study of biological and economic systems is a study of the ecological and evolutionary processes, and the resulting mechanisms, that act at different levels of organization and result in cohesive dynamics \cite{Levin2002}.
% 
Key challenges are to disentangle the necessary and sufficient elemental processes, and understand their couplings. 

Adressing those challenges, my work aimed at advancing our quantitative understanding of how ecological and evolutionary processes, and their interplay, shape the dynamics of biological and economic systems. In particular, this thesis contributed to 
% 
\begin{mylisti}
    \item a fundamental undersanding of the role of eco-evolutionary processes in shaping the dynamics of biological populations structured in complex landscapes \cite{chap1},
    \item the quantification of the effect of eco-evolutionary processes in economic systems \cref{chap3},
    \item methodological advances in the forward and inverse modelling of eco-evolutionary dynamics in biological and economic systems \cref{chap1,chap2,chap4}.
\end{mylisti}

In the following, I discuss the chapters of my thesis collectively within the broader context of our fundamental understanding and the modelling paradigm of the dynamics of biological and economic systems.

\subsection{Advances in the fundamental understanding of biological and eocnomic systems}

\subsubsection{Linking processes to patterns}
Spatial patterns of biodiversity result from microscopic processes acting upon individual organisms \cite{Champagnat2006}.
% \chapi contributed to advance our understanding on how eco-evolutionary processes and population structure influence population dynamics and phenotypic evolution in biological systems.
% 
Mutations result in the process of genetic drift, which promotes stochastic variations in the allelic proportions and phenotypes of biological populations \cite{XXX}. In spatially structured populations, this results in turn to "neutral differentiation", where spatially separated populations are inevitably characterised by differentiated alleles and traits \cite{XXX}. 
% 
The process of dispersal tends to reduce neutral differentiation, and this effect is modulated by landscape connectivity \cite{Wright1943,McRae2006,McRae2007} through the mechanism of "isolation by limited dispersal" \cite{Orsini2013}. By increasing the dispersal ability of organisms, landscape connectivity decreases neutral differentiation.
% 
When landscapes present heterogeneous habitats, natural selection can supplement the effect of genetic drift and increase the sole effect of stochasticity on differentiation. Under this scenario, local environment conditions select individuals with traits that provide them higher fitness \cite{XXX}. At the population level, this results in populations adapting to their local environment, a mechanism coined "local adaptation" \cite{Kawecki2004} and resulting in patterns of "adaptive differentiation". 
% 
% This results in adaptive differentiation, and is regarded as one of the most important factors govening species richness gradients \cite{Kawecki2004}.
% 
Adaptive differentiation is hindered by dispersal, which prevents local adaptation by bringing maladapted individuals, that destabilise the evolution of traits towards the optimal.
% 
% confounded by genetic drift, opposed by natural selection due to tempoeral envionrmental variability, and constrained by loack of genetic variation of by the genetic architecture of underlying traits \cite{Kawecki2004}. TODO: this may go in the perspective and limitations
% 
% Specifically, ref. \cite{Mirrahimi2020} presents a condition for local adaptation, where populations can locally adapt if the dispersal intensity $m$ is below a certain threshold involving the strength of natural selection $s$ and the strnegth of habitat heterogeneity $\theta$,
% \begin{equation}\label{eq:mirr_disp}
%     m < 2s\theta^2.
% \end{equation}
% 
While adaptive differentiation concerns traits under selection, it undirectly affects the differentiation of neutral traits, that are co-evolving with traits under selection through linkages \cite{XXX}. This results in the mechanism of "isolation by adaptation", where habitat heterogeneity, rather than landscape connecitivity, increases neutral differentiation \cite{nosil2008}. 
% 
Simple mechanisms resulting in neutral and adaptive differentiation are identified, but how they are modulated by eco-evolutionary feedbacks and landscape complexity is unclear. %TODO: unclear

In \chapi, I demonstrate a novel mechanism, involving the ecological process of competition for resources, that considerably affects neutral differentiation. Through the creation of unbalanced migration fluxes which affect the intensity of competition, heterogeneity in connectivity reduces gene flow and reinforces neutral differentiation. Through the accumulation of incompatibilities over time \cite{Dobhsanski}, this mechanism could lead to speciation over time, and contribute to the high diversification in mountain regions \cite{Rahbek}.

I also investigate the mechanism of local adaptation and how it results in adaptive differenetion in complex landscapes, where habitats are arranged in a realistic fashion. My results show that the complexity of habitat spatial distribution can be reduced to a measure of habitat spatial auto-correlation coined the "habitat assortativity" and denoted by $r_\Theta$. Landscapes characterised by a high $r_\Theta$ systematically support populations that are locally better adapted than in landscape with low $r_\Theta$, resulting in higher adaptive differentiation. Specifically, I provide an analytical condition for local adaptation that sheds light on how it relates to dispersal intensity, selection strength, habitat heterogneity, and $r_\Theta$.

Because $r_\Theta$ affects local adaptation, is must affect neutral differentiation throuh the mechanism of isolation by adapation. Closing the loop, I demonstrate that $r_\Theta$ affects population differentiation through two antagonistic effects. By favoring local adaptation, it promotes isolation by adaptation, therefore increasing neutral differentiation. Nonetheless, it also favors gene flow within clusters of similar environmental conditions, decreasing isolation by limited dispersal. This complex feedback is essential to understand population differentiation in comlex landscapes.

Overall, \chapi links fundamental mechanisms involved in the phenotypic differentiation of populations to eco-evolutionary feedbacks and complex population structures. % It extends on recent work including the interplay between ecological and evolutionary processes, and frequency dependence, hihglighting non-trivial emergent properties with large consequences on emergent patterns.

\subsubsection{Linking patterns to processes}

The processes that determine the dynamics of economic systems are unclear. 
% 
Exogenous drivers, such as technological change \cite{XXX}, economic institutions \cite{XXX}, and production costs \cite{Boschma2005a} have been proposed, but 
























% \subsection{Blending ML and mechanistic models to learn from data}

% \subsection{Quantitative support for eco-evolutionary processes in economic systems}

% \subsection{Novel methods for the modelling of complex adaptive systems}


% \section{Limitations}

% \subsection{Limitation of PDE methods}
% \cite{Akesson2021} : PDE methods are probably not as adapted as trait based ODEs. Those simpler models can already address important questions regarding climate change.
% \cite{Tacchella2018}: In many cases, not only in economics, theoretical modelling and forecasting are not tightly related5
% . Most of the modelling efforts are
% in the direction of oversimplified representations that aim only at understanding the potential effects of a single variable, or of a lim- ited set of them, in a controlled setting

% \section{Conclusion}
% In light of the results, XXX.
% % 
% We expect XXX.

% %%
% Consequently, this thesis contributed to a better understanding of XXX.
% % 
% While recent studies have underlined the need to account for XXX, we XX.

% \section{Perspectives}

% \subsection{Toward a continuum between ML and mechanistic models}

% \subsection{Evolutionary biology to undertsand the economic patterns}


% \subsubsection*{An urgent need for better understanding and model eco-evolutionary dynamics}
% The effect of direct anthropogenic pressure, together with climate change, is rapidly affecting ecosystems \cite{Ellis2011,Midgley2019}. Ecosystems are approaching state shifts \cite{Barnosky2011,Barnosky2012,Midgley2019}, which in turn will greatly affect human societies \cite{Mooney2009}.
% %
% %% Constatation of system state shift
% % Current extinction rates are higher than would be expected from the fossil record \cite{Barnosky2011}. %Based on habitat models, \cite{Midgley2019} predicts, on the basis of mid-range climate-warming scenarios for 2050, that 15\% to 37\% of species would be committed to extinction. 
% % 
% %%
% While there is a general agreement that anthropogenic pressure and climate change will have a negative effect on the biosphere \cite{fischlin2007ecosystems}, their precise effect on ecosytem dynamics is unclear \cite{Norberg2012}. In particular, the answer to how species will adapt to increasing temperatures is uncertain due to our lack of understanding of eco-evolutionary feedbacks \cite{Norberg2012}. 
% % 
% %For instance, with global warming, species are likey to shift towards higher elevations and higher latitudes \cite{Chen2011}. Because the speed of range shifts differ between different ecological groups, climate change is expected to modify the current organization of trophic interactions \cite{Descombes2020}, affecting ecosystem functioning.
% %
% %% 
% Current projections of ecosystem states, such as \cite{Midgley2019},
% are based on habitat models, where species habitats are deducted from species occurence data, and are reprojected it given environmental predictors.
% % 
% Such approaches miss the processes of ecological interactions, evolutionary change and species dispersal \cite{Pearson2003}, that are expected to play a critical role in the evolution of the biosphere in the coming decades \cite{Norberg2012}.
% % 
% In order to mitigate the consequences of human development, it is of utmost urgency to better understand eco-evolutionary feedbacks \cite{Norberg2012}, and develop mechanistic models embedding this knowledge \cite{Urban2016}. This will in turn provide more reliable forecasts of ecosystem states \cite{Clark2001}, to help designing adequate management of ecosystem services \cite{Urban2016}.

% % \subsubsection*{Endogenous forces in economic systems}
% % Traditional approaches to economics assume the rationality of economic agents \cite{XXX}. Economic dynamics
% % % 
% % In contrast, evolutionary economics 