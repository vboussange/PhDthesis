% !TEX root = ../my-thesis.tex
%
\chapter{Discussion}
\label{sec:conclusion}

\cleanchapterquote{Les données pertinentes détiennent les réponses}{French anagram}{}


\section{Contributions}
The goal of this thesis was to provide a better understanding on xxx, and to give new insights into xxx. In particular, this thesis contributed to a better understanding of (i) , (ii) and (iii).

\subsection{Interplay between spatial structure and eco-evolutionary mechanisms}

\subsection{Blending ML and mechanistic models to learn from data}

\subsection{Quantitative support for eco-evolutionary processes in economic systems}

\subsection{Novel methods for the modelling of complex adaptive systems}


\section{Limitations}

\subsection{Limitation of PDE methods}
\cite{Akesson2021} : PDE methods are probably not as adapted as trait based ODEs. Those simpler models can already address important questions regarding climate change.
\cite{Tacchella2018}: In many cases, not only in economics, theoretical modelling and forecasting are not tightly related5
. Most of the modelling efforts are
in the direction of oversimplified representations that aim only at understanding the potential effects of a single variable, or of a lim- ited set of them, in a controlled setting

\section{Conclusion}
In light of the results, XXX.
% 
We expect XXX.

%%
Consequently, this thesis contributed to a better understanding of XXX.
% 
While recent studies have underlined the need to account for XXX, we XX.

\section{Perspectives}

\subsection{Toward a continuum between ML and mechanistic models}

\subsection{Evolutionary biology to undertsand the economic patterns}


\subsubsection*{An urgent need for better understanding and model eco-evolutionary dynamics}
The effect of direct anthropogenic pressure, together with climate change, is rapidly affecting ecosystems \cite{Ellis2011,Midgley2019}. Ecosystems are approaching state shifts \cite{Barnosky2011,Barnosky2012,Midgley2019}, which in turn will greatly affect human societies \cite{Mooney2009}.
%
%% Constatation of system state shift
% Current extinction rates are higher than would be expected from the fossil record \cite{Barnosky2011}. %Based on habitat models, \cite{Midgley2019} predicts, on the basis of mid-range climate-warming scenarios for 2050, that 15\% to 37\% of species would be committed to extinction. 
% 
%%
While there is a general agreement that anthropogenic pressure and climate change will have a negative effect on the biosphere \cite{fischlin2007ecosystems}, their precise effect on ecosytem dynamics is unclear \cite{Norberg2012}. In particular, the answer to how species will adapt to increasing temperatures is uncertain due to our lack of understanding of eco-evolutionary feedbacks \cite{Norberg2012}. 
% 
%For instance, with global warming, species are likey to shift towards higher elevations and higher latitudes \cite{Chen2011}. Because the speed of range shifts differ between different ecological groups, climate change is expected to modify the current organization of trophic interactions \cite{Descombes2020}, affecting ecosystem functioning.
%
%% 
Current projections of ecosystem states, such as \cite{Midgley2019},
are based on habitat models, where species habitats are deducted from species occurence data, and are reprojected it given environmental predictors.
% 
Such approaches miss the processes of ecological interactions, evolutionary change and species dispersal \cite{Pearson2003}, that are expected to play a critical role in the evolution of the biosphere in the coming decades \cite{Norberg2012}.
% 
In order to mitigate the consequences of human development, it is of utmost urgency to better understand eco-evolutionary feedbacks \cite{Norberg2012}, and develop mechanistic models embedding this knowledge \cite{Urban2016}. This will in turn provide more reliable forecasts of ecosystem states \cite{Clark2001}, to help designing adequate management of ecosystem services \cite{Urban2016}.

% \subsubsection*{Endogenous forces in economic systems}
% Traditional approaches to economics assume the rationality of economic agents \cite{XXX}. Economic dynamics
% % 
% In contrast, evolutionary economics 